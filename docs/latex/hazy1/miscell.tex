\chapter{MISCELLANEOUS COMMANDS}
\label{sec:MiscellaneousCommands}
% !TEX root = hazy1.tex

\section{Overview}

This Chapter describes commands that are used to disable physical
processes, change the code's internal behavior, or to take care of
housekeeping activities.
These are not usually used.

\section{Introduction to the compile commands}

The following subsections describe how to set up and then compile some
external data files that are needed by the code.
In most cases, the stellar atmosphere
files must be compiled if you wish to use some of the
\cdCommand{table star} emergent radiation fields.
Most of the other files are included in the code's distribution and only
need to be compiled if you change the code.

The files produced by the compilation process must be accessible to the
code when it is executed from other directories.
For this the files need to be somewhere on the search path.
This is most easily done
if you place the compiled files in
the code's \cdFilename{data} directory.
See Section~\ref{sec:ReadingWritingFiles} for further options.

The \cdCommand{compile} command will be executed after
the command parser is finished
examining the input deck.
You must specify a complete model to get past
the parser's validation of the input stream.
The easiest way to do this
is to also include the \cdCommand{test} command
in addition to
\cdCommand{compile} commands.

The code stops after executing the first \cdCommand{compile}
command it comes across.
It will not do more than one compilation in a single run.

\section{Compile stars}

Kevin Volk incorporated several large grids of stellar atmosphere
continua
into \Cloudy\ in the 1990's.
Peter van Hoof has generalized the treatment
and made several extensions.
The \cdCommand{table star} commands use these atmospheres.
The data files can be very large and ``direct access'' is used to read these
files quickly.
The result is that the final files are not portable although
the code used to read or write them is.
The process of converting the
stellar atmosphere files from their original format into a form that can
be read by \Cloudy\ is referred to as compiling the stellar atmospheres.

The \cdCommand{table star} commands will only function
if the atmosphere files
are installed as described on the code's web site.
This is only done once
while installing \Cloudy\ although it will have to be
done again if you ever
change the continuum energy mesh.
This step does not need to be done if
you don't want to use these stellar atmospheres.

Full instructions on how to download and install the stellar continua
are given at
\href{http://wiki.nublado.org/wiki/StellarAtmospheres}{wiki.nublado.org/wiki/StellarAtmospheres}.

\begin{shaded}
\section{Compile opacities}

\cdTerm{N.B.} This command does not now function and may be removed.

When the code is initialized it spends some time evaluating numerical
fits to the needed opacities.
This initialization time can be saved if
the opacities are compiled and the resulting file placed on the path.
To
do this, execute the code and enter only the command
\cdCommand{compile opacities}.
The code will generate a binary file named \cdCommand{opacity.opc}
containing the needed opacities and array indices.

\cdTerm{N.B.}  It is not really necessary to compile the opacities---
the code
will generate them when it starts up if the file does not exist.
This may
actually slow down the calculation if you are using a fast computer on a
slow network.
\end{shaded}

\section{Compile recombination coefficients [H-like; He-like]}

The code reads in a table of recombination coefficients for the hydrogen and
helium isoelectronic sequences when it initializes.
This command regenerates that file.
It was
introduced by Ryan Porter.
One of the iso-electronic sequences must be
specified with either keyword \cdCommand{H-like} or \cdCommand{He-like}.

\section{Compile grains}
\label{sec:CompileGrains}

This command prepares the grain opacity files that are used by the
\cdCommand{grains} command.
This does not need to be done if you only want to use
the types of grains included in the data distribution.
You can create other species, with their own size
distribution and refractive indices, by compiling them with this command.
It is also necessary to compile the grains if you change the code's
continuum energy mesh.

This command uses a spherical Mie code originally developed by Peter
G. Martin using code written by \citet{Hansen1974}, and implemented
into \Cloudy\ by Peter van Hoof, to generate sets of grain opacities from
a description of their size distribution and grain material optical
properties (\citealp{VanHoof2004}).
The set of grain opacities created by
this \cdCommand{compile} command are then used to
compute temperature, charge, drift
velocity, and emitted spectrum, for each bin within the
grain-size distribution.
Section~\ref{grain:compile} describes how to create new grain opacity files.

A grain model is specified by the optical properties of the material
and by a description of the distribution of grain sizes.
The optical
properties are given by a file, with a name ending in ``\cdFilename{.rfi}'',
giving the
refractive indices as function of photon energy for the full energy range
considered by the code.
The grain-size distribution is specified in a
separate data file with a name ending in ``\cdFilename{.szd}''.
Sample refractive index
and size distribution files are included in the data directory or you can
create your own.
In all cases the \cdCommand{compile} command combines the refractive
index data with the size distribution to produce an opacity file
(with a name ending in ``\cdFilename{.opc}'').

Section~\ref{grain:appendix} includes far
more information and should be consulted.

\subsection{Compile grains}

With no options this compiles the minimum number of grain types needed
for the \cdCommand{grains} command to function.
Examine the output to check
for problems.

\subsection{Compile grain 10 bins, [filename, ism graphite]}

This version of the command produces opacities of a single grain species
and is used to create new types of grains.
Both the optical properties
and size distribution must be specified.
This is done by giving keywords
(to use one of the standard files located in the data distribution)
or a filename between double quotes (to read properties for a new species).

The number on the command line specifies the number of
grain size bins to compute.
If no number appears then 10 is set by default.

\cdTerm{Keywords for standard data sets}
One of a set of keywords may be used to
specify refractive index files and grain size distributions.
The keyword is usually the
name of a file included in the \Cloudy\ distribution
without the filename extension.

A grain-size
distribution may be specified with one of the keywords given in
Table~\ref{tab:GrainSizeDistributionKeywords}.
The first three are single-sized distribution functions that are mainly
used for testing the code.
For single-sized grains the supplied number
of cells is ignored since one cell is always used.
These files have names
that end in ``\cdFilename{.szd}''.

\begin{table}
\centering
\caption{Grain keywords for size distribution (*.szd files)}
\begin{tabular}{ll}
\hline
Keyword& Size distribution\\
0m010& 0.01 $\mu$m\\
0m100& 0.1 $\mu$m\\
1m000& 1 $\mu$m\\
ab08& \citealp{Abel2008}\\
c15& Small PAH, 15 C atoms\\
c120& Large PAH, 120 C atoms\\
ism& ISM distribution, \citealp{Mathis1977}\\
orion& Orion distribution, \citealp{Baldwin1991}\\
\hline
\label{tab:GrainSizeDistributionKeywords}
\end{tabular}
\end{table}

The keywords given in Table \ref{tab:GrainRefractiveIndexKeywords}
may be used to specify a file containing
refractive index information.
These files have names that end in
``\cdFilename{.rfi}''.

\begin{table}
\centering
\caption{Refractive index data files in the \protect\Cloudy\ distribution}
\begin{tabular}{llll}
\hline
Filename & Keyword & Grain type & Reference\\
\hline
\cdFilename{ac1-amcarb.rfi} & ac1-amcarb   & amorphous carbon      & \citealp{Rouleau1991} \\
\cdFilename{be1-amcarb.rfi} & be1-amcarb   & amorphous carbon      & \citealp{Rouleau1991} \\
\cdFilename{gdraine.rfi}    & ---          & graphite              & \citealp{Laor1993}    \\
\cdFilename{gdraine03.rfi}  & ---          & graphite              & \citealp{Draine2003}  \\
\cdFilename{graphite.rfi}   & graphite     & graphite              & \citealp{Martin1991}  \\
\cdFilename{grey.rfi}       & grey, gray   & grey grain            & ---                   \\
\cdFilename{pah1.rfi}       & pah          & PAH                   & Volk, private comm.   \\
\cdFilename{ph2c.rfi}       & ---          & charged PAH           & \citealp{Li2001}      \\
\cdFilename{ph2n.rfi}       & ---          & neutral PAH           & \citealp{Li2001}      \\
\cdFilename{ph3c.rfi}       & ---          & charged PAH           & \citealp{Draine2007}  \\
\cdFilename{ph3n.rfi}       & ---          & neutral PAH           & \citealp{Draine2007}  \\
\cdFilename{sdraine.rfi}    & ---          & astronomical silicate & \citealp{Laor1993}    \\
\cdFilename{sdraine03.rfi}  & ---          & astronomical silicate & \citealp{Draine2003}  \\
\cdFilename{sic.rfi}        & sic          & $\alpha$-SiC          & \citealp{Laor1993}    \\
\cdFilename{silicate.rfi}   & silicate     & astronomical silicate & \citealp{Martin1991}  \\
\cdFilename{vacuum.rfi}     & ---          & vacuum                & ---                   \\
\hline
\label{tab:GrainRefractiveIndexKeywords}
\end{tabular}
\end{table}

\cdTerm{Creating new data files}  If none of the keywords
listed in the tables
occur then the name of the file must be given explicitly.
This may be a user-generated file.
A pair of double quotes must
surround the filename, as in ``\cdFilename{name.rfi}''.
Names ending in
``\cdFilename{.rfi}'' are
refractive index files while those ending in ``\cdFilename{.szd}''
are size distribution files.
Filenames and keywords can be mixed on a single line as shown in
the examples below.

\subsection{Some caveats}

\begin{itemize}

\item
  If you run the code in another directory than where you compiled the grains,
  you must make that the files are moved to a location on the search path.
  See Section~\ref{sec:ReadingWritingFiles} for details.

\item
The grains will need to be recompiled if the energy mesh of the radiation
field is changed. 
The input script \cdFilename{compilegrains.in} in the \cdFilename{data}  
directory will do this.

\end{itemize}

\subsection{Examples}
\begin{verbatim}
# compile all grain types
compile grains

# only ism graphite
compile grains ism graphite

# use 20 bins for a silicate
# with Orion size distribution
compile grains 20 Orion silicate

# the grey.rfi file is the one read
# with the key grey, so the following
# is equivalent to grey ism
compile grains "grey.rfi" ism

# explicitly request the graphite.rfi
# (default with graphite) refractive index
# file, and the Orion size distribution
compile grains "graphite.rfi"  "orion.szd"

# these refractive index files are also
# present in the standard distribution
compile grains "gdraine.rfi" ism # graphite, Laor & Draine (1993)
compile grains "gdraine03.rfi" ism # graphite, Draine (2003)
compile grains "sdraine.rfi" ism # silicate, Laor & Draine (1993)
compile grains "sdraine03.rfi" ism # silicate, Draine (2003)
compile grains "ph2n.rfi" ab08 # neutral PAH, Li and Draine (2001)
compile grains "ph2c.rfi" ab08 # charged PAH, Li and Draine (2001)
compile grains "ph3n.rfi" ab08 # neutral PAH, Draine and Li (2007)
compile grains "ph3c.rfi" ab08 # charged PAH, Draine and Li (2007)
\end{verbatim}

\section{Crash [keywords]}

This command is intended to test that the code handles various error
conditions in the desired manner. These include floating point exceptions, but
also other errors like array bounds violations or the use of uninitialized
variables. Note that not all error conditions will be caught in a standard
setup. Array bounds violations and the use of uninitialized variables will not
be caught by default since that would slow down the code too much. One of the
following keywords must appear.

The IEEE standard for floating point arithmetic says to not throw an
exception when a floating-point exception occurs.
Instead the result is
set the result to \cdCommand{+/-inf} or \cdCommand{NaN} (not a number)
and have the calculation continue.
The code will only crash if this is explicitly requested by
setting compiler options or enabling traps on the CPU.
The source includes
logic that will do this on most platforms.
\Cloudy\ should crash on division
by zero, overflow, or invalid operation.
The \cdCommand{zero},
\cdCommand{overflow}, and \cdCommand{NaN} options
on the \cdCommand{crash} command confirms this.

\subsection{crash zero}

This command causes the code to divide a positive number by zero.  This
causes a division by zero exception.

\subsection{crash overflow}

This command causes the code to divide a very large number by a very
small number.  The result will overflow.

\subsection{crash long overflow}

This command causes the code to convert a floating number that
is larger than the largest long int into a long int.  Some machines may
throw an exception in this case (mine does not).

\subsection{crash NaN}

This command will divide zero by zero.  This will cause an invalid operation
exception.

\subsection{crash setnan [float, double]}

This command will test that the routine \cdTerm{set\_NaN} sets
variables to NaN.  A variable is set
to NaN with these routines and is then used in a multiplication.

\subsection{crash isnan [float, double]}

This command will test that the routine \cdTerm{is\_nan} correctly
determines whether a variable is NaN.  A variable is set to NaN with
\cdTerm{set\_nan}
then is asserted not to be NaN.  The monitor should fail.

\subsection{crash segfault}

This command causes the code to dereference a \cdTerm{NULL} pointer. This will
trigger a segmentation violation (segfault) and should always cause the code to crash.

\subsection{crash assert}

This command causes the code to assert that a positive number is less
than zero.  This should cause an exception when the code is compiled in
debug mode, but will have no effect when the optimization is set to a high
level or if the flag -DNDEBUG is included with the compiler options.  If
the code does indeed crash with a failed assert then it may request that
the output be posted on the web site.  This is normal and only shows that
the asserts are working properly.

\subsection{crash bounds}

This command will cause an array to be evaluated with an index that
exceeds the array bounds.  In some cases the code may crash when this occurs.

There are multiple types of arrays within the code and keywords say which
one to test.
Allocated arrays are created on the heap by using
\cdTerm{new}, \cdTerm{vector}, or \cdTerm{valarray}.
Bounds errors for these types of arrays can be trapped with the compiler or
external tools such as \cdTerm{valgrind} (http://valgrind.org)
and \cdTerm{Purify}.
Declared arrays are explicitly given a size when the variable
is declared in the source, as in
\begin{verbatim}
double c[5];
array<double,5> c;
\end{verbatim}
Modern versions of the GNU and LLVM C++ compilers have options to check that limits are not
exceeded on a declared array.
Finally, the \emph{multi\_arr}, \emph{flex\_arr}, and \emph{avx\_ptr} classes were introduced
after the code migrated to C++ and can check on array bounds if the macro
\cdCommand{BOUNDS\_CHECK} is set when the code is compiled.
This is done with the
compiler option \cdCommand{-DBOUNDS\_CHECK} on most compilers.

A keyword specifying the type of array to test must appear.
The keyword
\cdCommand{static} will use
a static array.
Either keyword \cdCommand{stack} or \cdCommand{auto} will use an array created
off the stack. The keyword \cdCommand{array} or \cdCommand{vector} will use an array created
using the corresponding STL class.
Code generated by a bounds-checking compiler should catch these errors.
The keyword
\cdCommand{multi} will
check an array created from the \cdVariable{multi\_arr} class.
The two keywords
\cdCommand{multi iter} will test an iterator in the
\cdVariable{multi\_arr} class.
These should crash if
the macro \cdCommand{BOUNDS\_CHECK} is set.
The \cdCommand{crash bounds multi
iter} command does
the same as the \cdCommand{crash bounds multi} command
except that it tests access
through iterators (which behave like pointers) rather
than direct access to the array.
Finally the keyword \cdCommand{avxptr} allows you can test an array created using the \cdVariable{avx\_ptr} class.
This should crash if
the macro \cdCommand{BOUNDS\_CHECK} is set.

The \cdCommand{crash bounds} command has an optional offset
to add to the array size \cdCommand{ARR\_SIZE}.
If $n$ is positive then the code will access
\cdCommand{array[ARR\_SIZE+n]}.
If it is negative then the code will access \cdCommand{array[n]}.
The default is zero,
that is, to access \cdCommand{array[ARR\_SIZE]}.
If either of two keywords, \cdCommand{low} or
\cdCommand{high}, occur with no number then the code will assume \cdCommand{n = 2} and check only the low or high bounds.

\subsection{crash exception}

This command will cause
the code to throw an exception.
The exception should be caught by the handler in the
main program.

\subsection{crash domain}

This command will call a vectorized math routine with invalid arguments. This
routine will then throw an exception. The exception should be caught by the
handler in the main program. This test will only be executed if AVX
vectorization has been enabled by the compiler.

\subsection{crash undefined}

This command multiplies a valid constant
by an undefined float created off the stack.
Special tools such as \cdRoutine{valgrind} can detect this and will report an error.
In normal execution, the fact that the variable is not initialized will not be
detected, though in some circumstances the multiplication may result in a floating
point exception.

\subsection{crash abort}

This command will cause the code to simulate a \Cloudy\ abort. This exception
should be caught and a message giving the reason for the abort should be
printed.

\subsection{crash grid}

This command will cause random grid points in a grid to fail due to a randomly
selected signal (e.g., a floating point exception) or a \Cloudy\ abort. On
average one in four grid points should fail. This command is intended to test
the robustness of the grid command against such failures and should never be
used in normal grid runs. The correct behavior of the code is that the grid
will finish normally. The \cdCommand{save grid} output should be complete and
correctly identify each failure. The output from other \cdCommand{save}
commands should also be complete and contain default (invalid) entries for
each of the failed grid points.

\section{Eden -2}
\label{sec:CommandEden}

This adds free electrons to the gas.
The argument is the log of the
electron density [cm$^{-3}$].
This command accepts the \cdCommand{vary} keyword.
It is mainly intended to test the
behavior of \Cloudy\ in the limit of very low Compton temperatures.
When
the color temperature $T_{color} \ll 10^4 \K$ the gas is almost
entirely neutral
and free electrons must be artificially added to test the Compton energy
exchange problem in the strict TE limit.
(Remember, charge conservation is a horrible thing to violate.)

Note:  This command adds an extra component of electrons.
The \cdCommand{set eden} commands,
described on page \pageref{sec:CommandSetEden}, set the electron density.
Both commands violate charge conservation.

\section{Fudge factors 4.59678 [12.3 34.5 958 \dots]}

The numbers appearing on the line can be communicated to any part of
the code by calling routine \cdRoutine{fudge}.
That routine has a single integer
argument that is an index to the array of numbers entered on this command
line.
A call to \cdRoutine{fudge}(0) would return the first number\footnote{In version 90 and before the array index was on the Fortran scale,
so fudge(1) returned the first number.  The code is now written in C++ and
the array index is interpreted on the C scale.  The first number is fudge[0].} and, in the
example given above, a call to \cdRoutine{fudge}(1) would return the value 12.3.
Up to ten numbers can be entered on the command line.

This command is used to pass numbers to temporary or trial parts of the
code.  Routine \cdRoutine{fudge} is a permanent part of \Cloudy\ and
a warning is given
at the end of the calculation if this function is evaluated.
The code will
stop if the index to the array of stored values exceeds
the number of values
entered.
Extra numbers are simply ignored.

The code will abort if \cdRoutine{fudge} is asked to return
a parameter that was not defined.
For instance if the command line reads
\begin{verbatim}
fudge 4.0
\end{verbatim}
then \cdTerm{fudge}(1) does not exist and the code will terminate.

\cdRoutine{fudge} will return the number of parameters entered
on the command line
if it is called with a negative argument.
The following code fragment shows
how to check on the number of parameters and then evaluate
\cdRoutine{fudge} if \cdRoutine{fudge} factors were entered.
\begin{verbatim}
factor = 1.;
if( fudge(-1) > 0 )
    factor = fudge(0);
\end{verbatim}

\section{Init [``c84.ini'', path]}

This reads a set of commands from an initialization file.
Frequently-used
commands can be stored in this file and easily accessed.

There is no limit to the number of commands that can be in this
initialization file.
An arbitrary number of \cdCommand{init} commands can appear in a single
input stream, as well as inside another initialization file.

The default name for the initialization file is \cdFilename{cloudy.ini}.  This will
be used if no double quotes occur on the command line.
Other file names
are specified by including a file name within a pair of double quotes as
in ``\cdFilename{special.ini}''.
The name can be any name allowed by the operating system in use.

The data directory includes the \cdFilename{ini} files that are used
in the test suite.
They were set up with particular applications in mind.
They may or may
not be appropriate for your purpose.
Please review the contents of these
files before using them.

\section{No \dots}

It is possible to disable physical processes as a test of their
importance.
If a physical process is turned off then a flag is set to
indicate that the treatment of a physical process has been disabled.

A warning will be printed at the end of the calculation as a reminder
that the results of the calculation are not to be trusted.
This warning
will not be printed if the four-character keyword (\cdCommand{OK}) appears on the command
line.
The parenthesis is part of this keyword.

\subsection{No advection [H-like, He-like, \dots]}

This turns off the effects of advection on the ionization, recombination
and energy of the gas.
Keywords make it possible to turn off these terms
for only certain parts of the calculation.
The \cdCommand{metals} option will turn
off the effects of advection on the ionization of any species not treated
as one of the iso-electronic sequences.
\cdCommand{Cooling} will turn off thermal effects of advection.

\subsection{No Auger effect}

This turns off the Auger effect.
Inner-shell ionizations will eject only one electron.

\subsection{No blends}

The code keeps a list multiplets and blends in the file \cdFilename{blends.ini}
in the data directory. This list is an init file that is automatically read by
the code when the input script is parsed (i.e., the code behaves implicitly as
if the command \cdCommand{init "blends.ini"} had been included at the end of
the script). Including the command \cdCommand{no blends} prevents the list from
being read. This can speed up the code when running large grids of single-zone
models.

\subsection{No buffering}
\label{sec:no_buffering}

Programs produce output by first writing into a buffer.
Information is written to disk once the buffer is full.
If a C++ program crashes
before this buffer is ``flushed'' the information within the buffer will
be lost.
This poses a problem if the printout generated just before the
crash is needed for debugging.
The \cdCommand{no buffering} command will redirect the
standard output to \cdFilename{stderr} which is not buffered.
\cdCommand{Save} commands also have
a \cdCommand{no buffering} option to turn off file buffering
for the files they create.

On most systems \cdFilename{stderr} can be directed to a file by redirection of file descriptor 2 as in the following example
(this assumes you are running a \cdCommand{bash} shell or something equivalent).
\begin{verbatim}
cloudy.exe < sim.in > sim.out 2> sim.err
\end{verbatim}
When the \cdCommand{-r} or \cdCommand{-p} command line flag is used, the
situation is different. In parallel runs, multiple output files would be open
at once and redirecting to \cdFilename{stderr} would create havoc. In this
case, the output file is closed and immediately reopened again in append mode.
Buffering can then safely be turned off. In this case you run the code as you
would normally.
\begin{verbatim}
cloudy.exe -r sim
\end{verbatim}

\subsection{No charge transfer}

This turns off all charge transfer interactions.

\subsection{No collisional ionization}

This turns off collisional ionization from ground states of all atoms
and ions.

\subsection{No Compton effect}

This command turns off Compton heating and cooling of free electrons
and Compton recoil ionization of bound electrons.
Electron scattering opacity is \emph{not} turned off.

\subsection{No CTHeat}

This turns off heating due to charge transfer.
\citet{KingdonFerland1999} describe the process.

\subsection{No diffuse line pumping}

The diffuse continuum produced by gas within the cloud is included as
a general line-excitation mechanism.
This turns it off and is useful as
a check on the importance of the process.
(This command is currently disabled. The code will issue 
a statement to that effect and exit if this command
is entered.  Pumping by local diffuse emission
is not included.  Pumping by diffuse emission from previous, downstream
zones is always included.)

\subsection{No FeII pumping}

This turns off \hi\ L$\alpha $ pumping of \feii.

\subsection{No file opacity}

The code can generate a file of stored opacities with the
\cdCommand{compile opacity} command.
This file will be used to generate the opacities in later
calculations.
The \cdCommand{no file opacity} command tells the code to ignore this
opacity file even if it exists.

\subsection{No fine structure line optical depths}

Fine structure lines, such as the $^3P\ 52,\ 88 \micron$ lines of [\oiii],
can become optically thick under certain high-luminosity conditions
(see, for example, \citealp{Rubin1983}).
They can absorb the incident continuum and be a significant
heating source for photodissociation regions
(\citealp{Tielens1985a}).
Radiative transfer effects, including stimulated and maser emission, are
fully treated for all lines.
This turns off optical depths and line transfer
for fine structure lines by setting the line opacity to zero.

\emph{This command does not turn off line heating}---which
will then be maximized
since the lines will remain optically thin.
The \cdCommand{no induced processes} command
turns continuum pumping off for all lines
and should be included to disable heating due to line absorption of the
continuum.

The line transfer arrays are permanently injured by the
\cdCommand{no fine structure} command.
Subsequent runs with the same core load in a grid will still have
the line optical depths disabled.

\subsection{No fine opacities}

The code uses a fine continuum and opacity grid to automatically account
for line overlap.
Figure 20 of \citet{Shaw2005} shows an example of this
fine mesh.
Multi-grid techniques are used to propagate this fine mesh back
onto the coarse mesh that is used to evaluate photo rates.
This fine grid
can be turned off using this command.

\subsection{No free free}

Free-free heating, cooling, and emission are turned off with this command.

\subsection{No grain [process]}

This series of commands disables various processes involving grains.
\begin{description}
\item[no grain electrons]   This turns off the effects of grain charging, either
negative or positive, on the overall charge balance in the gas.  This
violates charge conservation.

\item[no grain gas collisional energy exchange]   Collisional energy exchange
between grains and the gas is treated as in \citet{VanHoof2004} and \citet{Abel2005}.   This processes acts to bring the grains and gas toward
the same temperature.  The two temperatures will be equal if collision energy
exchange is fast.  Often the gas is hotter than the grains in the \hplus\ region
and this process heats the grains and cools the gas.  In fully molecular
regions typically the reverse is true as the remaining continuum is very
ineffective in heating the gas, but can still heat the grains. This command
disables this physics.

\item[no grain molecules]  Condensation of molecules onto cold grain surfaces
was added by Nick Abel and is treated using the formalism outlined by
\citet{Hasegawa1992} and \citet{Hasegawa1993}.  This turns off that
process.

\item[no grain neutralization]  Ions can become neutralized or ionized, by
several stages of ionization, following a collision with a grain.
The treatment of these collisions is described in \citet{VanHoof2004}.
It is partially based on \citet{Draine1987}'s description of
Coulomb interactions between the grains and gas.
The effects of grain neutralization upon the ionization
balance of the gas are turned off with this command although grain charging
is still computed.  This is not self consistent.

\item[no grain physics]  This turns off all grain physics.  The only remaining
effect of grains is their continuous opacity.  The resulting simulation
will be much faster but incorrect.  This is only intended to provide a method
to do quick and dirty calculations. The spectrum emitted by the grains is
not calculated so this command violates energy conservation.

\item[no grain qheat]   This turns off quantum heating for all grain species.

\item[no grain x-ray treatment]  This turns off the
extensive treatment of inner
shell X-ray processes within grains.
This makes the X-ray physics more
like \citet{Weingartner2001a}.
\end{description}

\subsection{No induced processes}

This turns off induced recombination and its cooling, stimulated
two-photon emission and absorption (\citealp{Bottorff2006}), continuum
fluorescent excitation, and stimulated emission of all lines.

\subsection{No ionization reevaluation}

This tells the code to not constantly reevaluate the ionization rates.
This option can only be set when constant density is assumed.

\subsection{No isotropic continua report}
\label{sec:no_isotropic_continua}

By default, the code reports the total outward flux emitted
by a cloud, including any attenuated isotropic continua.
This is effectively what telescopes observe in the optical
through X-ray bands.
However, in the infrared through radio bands, it is customary
(if not automated) to remove any background radiation with a
variety of methods (e.g., source or frequency switching).
This is part of the {\em Herschel} pipeline, for instance.

\par
This command removes isotropic continua from the {\it reported}
continua and band fluxes, to ease comparison against observations
at longer than IR wavelengths.
This only affects the output:
Isotropic sources of radiation are still included in the physics
problem \Cloudy\ solves.

\par
This command affects the output produced by some \cdCommand{save continuum} commands,
as described in section \ref{sec:save_cont_no_isotropic_option},
as well as the fluxes through the bands described in
Section~\ref{Hazy2-sec:Radiation-field-integrated-over-wavelengths}
\cdSectionTitle{\refname{Hazy2-sec:Radiation-field-integrated-over-wavelengths}},
and defined in the file ``data/continuum\_bands.ini''.
It has no effect on the continua reported in time-dependent
integrations.

\subsection{No line transfer}

This turns off line transfer for all lines except the
\la\ transitions
of the species treated with the iso-sequences.
This is the approximation
made in most codes designed to consider classical nebulae and may be
appropriate in some cases.
This command also turns off line radiation pressure.


\subsection{No lines isotropic continuum subtraction}
\label{sec:no-lines-isotropic-continuum-subtraction}

By default, reported line intensities are modified by the diminution factor
due to pumping by isotropic continuum radiation, most notably the CMB.
The physics of the diminution is explained in detail in \citet{Chatzikos2013},
in connection to the escape probability formalism.
The main effect is best understood in a two-level atom, where isotropic
continuua sustain an upper level population that reduces the number of
spontaneous deexcitations relative to the non-isotropic pumping case,
which is driven by collisions.
This diminution factor has been computed in connection to hyperfine structure
lines by \citet{DCruz1998}, and has been invoked ad hoc by \citet{Goldreich1974}.

\par
With this command the correction is not applied.
This command affects the output of
\cdCommand{save lines cumulative},
\cdCommand{save lines intensity},
and \cdCommand{save linelist}.


\subsection{No Lya 21cm pumping}

Pumping of the \hi\ 21 cm line by L$\alpha $ is treated
as in \citet{Deguchi1985}.
This turns off the process.
All three keywords must match.

\subsection{No OTS [options]}
\begin{description}
\item[no Lya OTS]  This turns off \hi\ \la\ OTS rates.

\item[no HeII OTS]  This turns off He~II \la\ and recombination continua OTS rates.

\item[no line OTS]  This turns off all line OTS rates.
\end{description}

\subsection{No level 2 lines}

This turns off the large block of Opacity Project (``level 2'') lines.
This should only be done in exploratory ``quick look'' calculations.
The lines should be included for final results.
These lines should not
be disabled at higher densities since they may carry a large fraction of
the cooling as the cloud approaches the black-body limit.
They are
important in UV photoexcitation of excited levels within
split ground terms like \oi* or \cii*.
Excited-state column densities will not be predicted if
the level 2 lines are disabled.

\subsection{No molecules [ heavy ]}

\Cloudy\ includes a chemistry network for predicting molecular abundances that
is described in \citet{Abel2005} and \citet{Roellig2007}. The \cdCommand{no
  molecules} command will completely turn off this network. If the keyword
\cdCommand{heavy} is also present on the command line, the chemistry network
will be severely limited by excluding any molecular species containing
elements heavier than hydrogen.

\subsection{No on the spot}

This turns on all ground-state recombination coefficients and turns off
ionization by helium resonance lines.
This last variable is then used to
deduce the ionization efficiency of lines and continua.

\subsection{No opacity reevaluation}

Opacities are normally reevaluated every time the cooling function is
reevaluated.
When this is set opacities are only evaluated one time per zone.

\subsection{No outward [options]  }
\begin{description}
\item[no outward lines]  This turns off the outward-only (\citealp{Tarter1967})
contribution by lines.

\item[no outward continuum]  This turns off the outward-only contribution by
continua.
\end{description}

\subsection{No photoionization}

This turns off photoionization of the ground states of all elements.
It is designed to test the code in the collisional-ionization equilibrium
limit.

This command can cause problems.
Several iso-sequences are treated as
full multi-level atoms.
This includes all processes that affect internal
excitations.
The level-populations solver may have difficulty in finding
a solution when photoionization is turned off with this command but the
gas is very weakly collisionally ionized.
Often internal continuum pumping
of the Lyman lines will dominate excitations.
When these radiative terms
overwhelm the collisional-ionization term the ionization rates out of the
ground state may underflow.
In this case the solver will find negative
level populations and the code will stop announcing a
catastrophic failure.
One solution is to turn off continuum pumping,
with the \cdCommand{no induced processes} command.
Only do this if the calculation fails.

\subsection{No radiation pressure}

This command turns radiation pressure completely off.
Radiation pressure
due to trapped lines will be counted in the total pressure when
the constant pressure option is used.
The default is for a constant-density model.
Radiation pressure is not included if constant gas pressure is specified.

\subsection{No recoil ionization}

This command turns off Compton recoil ionization.
Compton heating and
cooling of free electrons is included,
but this is the only remaining thermal
effect of electron scattering.
Bound-electron scattering opacity is still
included when this command is issued.

\subsection{No scattering escape}

Turn off line escape following scattering off a free electron.

\subsection{No scattering opacity}

See page \pageref{sec:CommandNoScatteringOpacity} above.

\subsection{No secondary ionizations}

This command will turn off the effects of knock-on supra-thermal
electrons.
Normally these are treated as in \citet{Spitzer1968},
\citet{Bergeron1971}, \citet{Shull1979}, \citet{Shull1985}, \citet{Xu1991},
and \citet{Dalgarno1999}.
This command
will make X-rays 100\% effective in generating heat and
produce no secondary
ionizations or L$\alpha $ excitations.

\subsection{No Stark broadening}

Stark broadening (important for densities larger than
$\sim 10^{10}\mathrm{cm}^{-3}$) is
treated for hydrogen lines using the formalism described by
\citet{Puetter1981}.
This turns Stark broadening off.

\subsection{No TePredictor}

The code tries to predict the temperature of the next zone for constant
density calculations.
This stops the predictions from being used.

\subsection{No static opacities}

This forces all opacities to be reevaluated constantly within each zone.
The default is to only evaluate minor opacities one time per zone.

\subsection{No three body recombination}

This turns off three-body recombination for the heavy elements that are
treated as equivalent two-level systems.
Three-body recombination for atoms
and ions of the isoelectronic sequences is turned off by turning off
collisional ionization (its time reversal) with the
\cdCommand{database [H-like or He-like] collisional ionization off} command.

\subsection{No times}

This turns off printout that would introduce differences between 
output files which have identical results.
The code will not report execution times or one-time notices about initializing databases. 
This allow exact text comparisons of results between models run at
different times or on different machines.


\subsection{No vary}

This command turns off the \cdCommand{vary} option set on
various optimization commands.

\section{Monitor commands}
\label{sec:CommandMonitor}

These tell the code a set of results that are expected.
At the end of
the calculation the expected and actual results are compared and an error
condition is returned if the two disagree by more than an uncertainty.
This set of commands is the foundation for automated testing of the code.
In Lexington the full test suite is computed every night and a log is kept
of all monitored results.
An email warning is sent if any monitor fail.
This insures the reliability of the code and guarantees that bugs are caught
almost as soon as they are introduced.

If the characters ``$<$'' or ``$>$'' appear on the
command line the expected result is taken as an upper or lower limit.
The default is to check for equality.

The relative error is the second optional number on the line.
If this
is not specified then a relative error of \ErrorDefault\ will be assumed.

\subsection{Monitored results in the main output}

The monitored quantities will be reported at the end of the main output, in the following format.
\begin{verbatim}
                   Label         line    computed     asserted Rel Err Set err
 ChkMonitor        H  1      4861.36A   -11.2479 =   -11.2530  -0.012   0.050
 ChkMonitor -      H  1      3889.07A     0.1084 =     0.1070  -0.013   0.050
 ChkMonitor        Ca B      3889.07A     0.1046 =     0.1050   0.004   0.050
 ChkMonitor -      He 1      5875.64A     0.1494 =     0.1470  -0.016   0.050
 ChkMonitor        Q(H)      4861.36A     0.8953 >     0.7800  -0.148   0.050
\end{verbatim}
The \cdCommand{save monitors} command, described in section \ref{sec:SaveMonitorsCommand},
can be used to save these in an external file.
 
\subsection{Monitor set error 0.1}

This provides a way to change the default relative error for all monitors
of physics-based quantities that occur later in the input stream.
The default error is normally the second optional number on each
\cdCommand{monitor} command.
The default is changed when this command is parsed so this should
be the first \cdCommand{monitor} command in the input stream if
you want to change the default error for all the monitors.

The majority of the monitors confirm quantities that are related to
predictions.
There are a special set of \cdCommand{monitor} commands which
track the code's performance.  
The \cdCommand{monitor itrzn} command is an example.
A default relative error of \ErrorDefaultPerformance\ is assumed for
performance monitors.
This default can be changed with the
\cdCommand{Monitor set performance error xxx} command.

\subsection{Monitor $\langle$subcommand$\rangle$ grid ``filename''}

For each of the commands discussed below you can add the keyword
\cdCommand{grid}. If this keyword appears, then a file name should be
supplied between double quotes. This file should contain a series of
quantities to be monitored. There must be one value for each grid point in a
grid calculation (the \cdCommand{grid} command is described elsewhere). The
numbers in the file have the same meaning as the number on the command line
when the \cdCommand{grid} keyword is not used. See the discussion for each
subcommand for further details. The file should not contain any other text,
but may contain comments starting with ``\#''. The numbers should be separated by whitespace and/or
newlines.

\subsection{Monitor Case~B [H-like, He-like]  element error}

The H and He recombination emission from lower density
photoionized clouds
that are optically thick in the Lyman continuum may be
close to Case~B values (AGN3).
This command uses stored tables of Case~B emission coefficients
to compare the code's computed emission with Case~B expectations.
Assuming that the cloud conditions are set up properly,
and that certain physical processes are disabled,
this provides a way to check the accuracy of models
of some atoms.

One of the iso-sequences, H-like or He-like, must be specified.
The default
relative error can be changed by specifying a number on the line.
The code
will compare Case~B with computed values for all lines that are included
in the emission-line output
and a botched monitor will be declared if any
of the code's predictions differ from Case~B predictions by more than this
error.
A line will be printed that gives the mean error for the Case~B
comparisons.
\begin{description}
\item[monitor Case~B H-like [element] error]  The H-like option will compare
the code's predictions with the \citet{Storey1995} results.  An element
name, any between hydrogen and oxygen, must also appear.

\item[monitor Case~B He-like error]  This will compare the predictions with Case~B predictions for He~I emission given by 
\citet{Porter.R12Improved-He-I-emissivities-in-the-case-B-approximation, Porter.R13-CaseB-erratum}.  
The Case~B values have the label ``+Col'' in the main emission-line printout. The
element name must still appear but currently only helium itself can be
tested.

\item[The range option]  The keyword \cdCommand{range}
tells the code to only check the
monitored values over a restricted wavelength range.
The wavelengths of
the lines are stored in Angstroms so the range must be given in
the same units.
This was added as a way to exclude the intensities of
unobservable MIR - FIR lines.  These lines come from higher levels in the
atom and generally have a larger error due to the finite size of the model.
\end{description}

\subsection{Monitor column ``species'' 18.9}

This checks the predicted column density for the molecular species
included in quotes.

An alternative form with unquoted keywords \cdCommand{CO},
\cdCommand{H-}, \cdCommand{H2}, \cdCommand{H2+}, \cdCommand{H3+},
\cdCommand{H2g}, \cdCommand{H2*}, \cdCommand{HeH+}, \cdCommand{O2},
\cdCommand{SiO}, \cdCommand{C2}, \cdCommand{C3}, \cdCommand{OH},
\cdCommand{CN}, \cdCommand{CH+}, or \cdCommand{CH} is available at
present but it is planned will be removed.  To use this form, the
labels must be surrounded by spaces.  It is important that the log of
the column density [cm$^{-2}$] appear \emph{after} the label because
of the 2 in \cdCommand{H2} or \cdCommand{O2}, etc.

The command accepts a \cdCommand{log} option to interpret
the error as a log.
The column density itself is always interpreted as a log.

For \htwo\ there is a \cdCommand{levels} option which tells the code
to return the column density in a specific $v$, $J$ level.
This only works when the large \htwo\ molecule
is included.
The numbers on the line must appear in the following order:
first the 2 in \cdCommand{H2}, then $v$,
and then $J$, and finally log~$N$(\htwo).
The ``wavelength'' of the
monitored quantity will be $100v + J$.

\subsection{Monitor csupra $-$17.09}

The secondary ionization rate is given by the variable \emph{csupra}.
This command checks the value of the secondary ionization rate.
The number is the log of the secondary ionization rate [s$^{-1}$].

\subsection{Monitor Ctot -12.5}

This checks that the local cooling rate [\ergpccmps ]
of the last zone is equal to the expected value.
The argument is the log of the cooling rate.

\subsection{Monitor departure coefficients  mean=1, error=0.04}

This confirms that the departure coefficients predicted by one of the
model atoms are correct.
The computed quantity is the mean departure coefficient for all levels
in the model atom and is the first number on the line.
The maximum allowed error from the mean is the second number on the line.

An alternative form for the command may be used to test against upper
and lower limits for the departure coeffient.
For example,\\
%
\cdCommand{monitor departure coefficients ``C[4]'' $>$ 0}\\
\cdCommand{monitor departure coefficients ``C[4]'' $<$ 1}\\
%
\noindent
may be used to test the usual case that the departure coefficient
is within the range 0--1.
No error is expected in this use case.

\begin{description}

\item[monitor departure coefficient ``C+4'']
If a species (in chemical notation) is given in quotes,
as \cdFilename{"O+6"} or \cdFilename{"H2O"},
then departure coefficients for that species will be checked.
The species designation may include a range of levels,
e.g., ``C+4[3:10]'', or a specific level, e.g., ``C+4[3]''.
The rules of Section~\ref{sec:SaveSpeciesLevelPopulations} apply.

\item[monitor departure coefficients, H-like helium, error=0.08]
The keyword
\cdCommand{H-like} tells the code that the departure coefficients predicted by an atom
of the hydrogenic isoelectronic sequence should be checked.  The name of
one of the elements must also appear on the command line.

If the keyword \cdCommand{excited} appears then the ground state will not be included.
The keyword \cdCommand{ZeroOK} allows departure coefficients of exactly zero to pass
the test.  This is useful in large grids of calculations where the species
being monitored actually has zero abundance at some points in the grid.

\item[monitor departure coefficients, He-like helium, error=0.08]
The keyword \cdCommand{He-like} tells the code to check the predictions
for the specified element of the helium isoelectronic sequence.
It operates like the H-like keyword, and accepts the \cdCommand{excited}
and \cdCommand{ZeroOK} keywords, as well.

\item[monitor departure coefficient, hminus, error=0.08]
The keyword \cdCommand{hminus} tells the code to check
the departure coefficient of H$^-$.
\end{description}

\subsection{Monitor depth 13.2}

This checks that the cloud's thickness [cm] is equal to the expected
value.
The argument is always the log of the depth.
The thickness is the
length between the illuminated and shielded faces of the cloud.

\subsection{Monitor eden 9}

The number is the log of the electron density [cm$^{-3}$] of the last zone.

\subsection{Monitor grain potential}

The grain bin number is the first number on the line.  Next comes the
grain potential.
This is linear eV and can be positive or negative.
The third number is the optional error.

\subsection{Monitor H2 grain rate -16.5}

The number is the total rate [cm$^{-3}$~s$^{-1}$] that \htwo\ forms on grain surfaces
for the last computed zone.
The number is a log if it is negative.

\subsection{Monitor H2 ortho/para ratio 2.02}

This checks the ratio of the ortho to para \htwo\ densities in the last
computed zone.
This is only meaningful if the large \htwo\ molecule has been
turned on.

\subsection{Monitor Htot -13.2}

This checks that the local heating rate [\ergpccmps ]
of the last zone is equal to the expected value.
The argument is the log of the heating rate.

\subsection{Monitor HHeicf -0.013}

This checks the helium---hydrogen ionization correction factor.
The number is the linear difference between the atomic fractions
of helium and hydrogen.

\subsection{Monitor ionization fraction oxygen 3 -3.45, error 0.1, weight =
radius}

This checks the mean fractional ionization of an ionization stage of
an element.
An element name must appear somewhere on the line.
The first
number is the ionization stage, 1 for the atom, 2 for the first ion, etc.
The second number is the expected ionization fraction.
It is interpreted
as a log if negative and as the linear ionization fraction if it is positive.
The average can be with respect to radius or volume.
The default is an
average weighted with radius.
If the keyword \cdCommand{volume} appears then the average
will be weighted over volume.

If the keyword \cdCommand{grid} appears then a file name should be supplied
between double quotes. This file should contain a series of ionization
fractions to be monitored. There must be one value for each grid point in a
grid calculation (the \cdCommand{grid} command is described below). The file
should not contain any other text, not even comments. The numbers should be
separated by whitespace and/or newlines.

An optional error may be supplied on the command line. The same error is
assumed for each grid point.

\subsection{Monitor molecular fraction H2 -3.45, error 0.1}

This checks the mean molecular fraction.
The label for a molecule,
currently only \htwo, must appear somewhere on the line.
The next number is
the expected molecular fraction.
If this number is $\le 0$ it is interpreted
as a log and as the linear ionization fraction if it is positive.
The
average can be with respect to radius or volume.
The default is an average
weighted with radius but if the keyword \cdCommand{volume} appears then the average
will be weighted over volume.

\subsection{Monitor mpi}

This checks that the code is running in MPI mode. The command is used
in the \cdFilename{mpi} test suite. It has no additional parameters.

\subsection{Monitor itrzn $<$ 3.5}

This performance monitor checks the convergence properties of a calculation.
The quantity
is the number of iterations required to converge each zone.

\subsection{Monitor line [Case B] ``H  1'' 4861.33 $<$ 1.01}

This checks the emission in a line.
The string that gives the line label must appear between
two double quotes.
The line wavelength is the first number after the label.
Both label and wavelength must appear exactly as they do in the output
produced by the \cdCommand{save line labels} command or
in the standard emission-line output. These two entries may optionally
be followed by additional information to disambiguate the line.
The information given in Section~\ref{sec:SpecifySpectralLines} applies
here as well, with the obvious exceptions that the line label should
not start in the first column, and that additional information is
required after the line information, as described below.

If the keywords \cdCommand{Case B} appear,
the code will compute the ratio of the intensity of the specified line
over the Case B intensity of the same line, which is interpolated from
the \citet{Storey1995} tables.
Compared to \cdCommand{monitor Case~B [H-like, He-like]},
which goes over all lines on the tables,
this command allows for individual lines to be monitored.

The sub-keyword is actually ``\cdCommand{line\_}'' to avoid
confusion with the keyword ``\cdCommand{linear}'' that appears
on some commands.
The command will not be recognized if the trailing space is missing.

The second number that appears in the command is the expected intensity of the line.
Its meaning is as follows:
%
\begin{itemize}

\item \emph{Relative intensities}

If the keywords \cdCommand{luminosity} and \cdCommand{intensity} do not appear,
the second number is interpreted as the emission line intensity relative to the
normalization line intensity.
It is always a {\it linear} quantity.

\item \emph{Absolute intensities}

If the keyword \cdCommand{luminosity} or \cdCommand{intensity} appears, then
the absolute luminosity [erg s$^{-1}$] or intensity [erg cm$^{-2}$~s$^{-1}$]
of the line is checked.
The monitored quantity is the {\it log} of the luminosity or intensity of
the transition, and not the intensity relative to the normalization line.
For example,
%
\begin{verbatim}
monitor line [luminosity intensity] "Q(H)" 4861 38.91
\end{verbatim}
%

\item \emph{Line types: intrinsic, emergent, and cumulative}

We predict four types of line intensities: intrinsic, emergent,
and their cumulative (time-integrated) counterparts, i.e.,
cumulative intrinsic, and cumulative emergent.
The first two are described in Hazy 2, Section
\ref{Hazy2-sec:LineIntensitiesDustyCloud}, 
\cdSectionTitle{\refname{Hazy2-sec:LineIntensitiesDustyCloud}}.
The cumulative intensities are integrated according to the
weight set by the \cdCommand{set cumulative} command, described
in full in Section \ref{sec:SetCumulativeCommand}.

In the absence of any of these keywords, the monitored intensity is
interpreted as \cdCommand{intrinsic}.
If only the keyword \cdCommand{cumulative} appears, the intensity is
interpreted as \cdCommand{cumulative intrinsic}.
Checks against the (instantaneous, non-integrated) emergent intensity
are issued with the \cdCommand{emergent} keyword, while checks against
the time-integrated emergent intensity are issued with the
\cdCommand{cumulative emergent} keywords.

\item \emph{Upper or lower limits}

Upper and lower limits may be requested by using the usual `$<$' and '$>$'
symbols, respectively, in front of the desired value, as in:
%
\begin{verbatim}
monitor line luminosity "Q(H)" 4861 < 38.91
\end{verbatim}

\end{itemize}

\subsection{Monitor niter $<$ 4}

This checks the number of iterations required in a calculation.
This is usually an upper limit.

\subsection{Monitor nothing 0}

This will enter a clean monitor into the output.
This is used for those
simulations in the test suite which have nothing to
monitor but are included
for general testing.

\subsection{Monitor nzone $<$ 135}

This checks how many zones were needed.
This is usually an upper limit.

\subsection{Monitor pressure error $<$ 0.01}

The number is the ratio of the standard deviation of the pressure to
its mean value.
It is interpreted as a log if it is negative.

\subsection{Monitor PRadMax 0.34}

The monitored quantity is the maximum ratio of radiation to gas pressure.

\subsection{Monitor temperature ``species'' 4.02 volume, error 0.01}

This checks the mean temperature of any species.
An ionic or molecular species is expected within quotes,
e.g., ``H'', ``He+2'', or ``CH2+''.
If the species is recognized, the code will compare the computed mean
temperature for that stage of ionization with the monitored value.
The number on the line is the temperature, interpreted as a log if
it is $\le 10$.
It will be linear if the keyword \cdCommand{linear} appears.
The last number is the relative error.

The code computes means weighted over radius and over volume.
If the keyword \cdCommand{volume} appears then the temperature will
be compared with the volume-weighted mean.
The default is weighting over radius.

If the keyword \cdCommand{grid} appears, then a file name should be supplied
between double quotes.
\textit{The file name must be given before the double-quoted species.}
The syntax is:
%
\begin{verbatim}
monitor temperature grid "InputFile.txt" "species" 4.02 volume, error 0.01
\end{verbatim}
%
This file should contain a series of temperatures to be monitored.
There must be one value for each grid point in a grid calculation
(the \cdCommand{grid} command is described elsewhere). The file should not contain
any other text, not even comments. The numbers should be separated by whitespace
and/or newlines.
An optional error may be supplied on the command line; the same error is
assumed for each grid point.
For example,
%
\begin{verbatim}
monitor temperature grid "ism_grid.dat" "H+" .05
\end{verbatim}
%

\subsection{Monitor 21cm temperature [mean, spin, optical] 50 error 0.1}

This monitors the temperature of the 21cm hyperfine transition of hydrogen.
The \cdCommand{mean} option tracks the kinetic temperature of the gas
weighted by hydrogen, \cdCommand{spin} tracks the spin temperature of the 21cm line,
while \cdCommand{optical} tracks the temperature of the gas computed from the ratio
of the optical depths in the \la{} and 21cm lines, see AGN3, section 5.5.

The first number is the temperature, interpreted as a log if
it is $\le 10$.
It will be linear if the keyword \cdCommand{linear} appears.
The last number is the relative error.

\subsection{Monitor grain temperature index 2, temperature  234}

If no element name appears but the keyword \cdCommand{grains}
does then the code
will compare the computed and monitored grain temperatures.
The first number
on the line is an index giving the grain type.
This index is from the order
in which the grains were specified in the input stream and
is given in the output.
The first grain that occurs in the input stream is number 1.
The
second number is the temperature and is interpreted as a
log if $\le 10$.
The
\cdCommand{linear} keyword forces smaller numbers to be
interpreted as linear quantities.
The optional error is the third number on the line.
The temperature is
always averaged over radius.

\subsection{Monitor temperature at face 11400K}

This checks the temperature at the illuminated face of the cloud.

\subsection{Monitor radius 18.2}

This checks that the outer radius [cm] of the computed model is equal
to the expected value.
The radius is the distance from the center of the
central object to the shielded face of the cloud.

\subsection{Monitor thickness 13.2}

This checks the thickness [cm] of the computed structure.
The argument
is always the log of the thickness.
The thickness is the length between
the illuminated and shielded faces.

\subsection{Monitor velocity 7.6 km/s}

This checks the final velocity of a wind model.
The quantity is the
expected final velocity in km s$^{-1}$.

\section{Set commands}

These commands change internal variables used by \Cloudy.
These are mainly
intended for testing and are not normally used.

\subsection{Set Lya 21cm [excitation, kinetic, constant]}
This changes the form of the Ly$\alpha$ source function $S_{\nu}$ at line center.
The shape of the source function has a major impact on the excitation
and resulting spin temperature of the 21 cm line.
Three options are available.  The first two are characterized by a Planck function
at one of the temperatures that are present, while the last shape assumes $S_{\nu} \sim constant$.

The Ly$\alpha$ excitation temperature, as set by the $n(2p)/n(1s)$ population ratio, is the default
and can be selected with the keyword \cdCommand{excitation}.
The gas kinetic temperature is used if the keyword \cdCommand{kinetic} appears.
A frequency-independent source function is assumed if the keyword \cdCommand{constant} appears.

\subsection{Set 12C13C -3.2 -- DEPRECATED}

This command has been deprecated.
Use \cdCommand{element carbon isotopes} instead.

\subsection{Set blend [ "Custom" ] [ 4622.23 ] [ quiet ]}

With this command you can define a custom blend of lines. In order to identify
the blend, you need to enter a label between double quotes and a wavelength.
If the label is omitted, it will default to ``Blnd''. If it is too long, it
will be truncated to \LabelLenMax\ characters. If the wavelength is omitted,
it will default to the $g_k A_{ki}$-averaged wavelength of the blend
components. The latter is only meaningful if the blend components are all from
the same ion, which in general need not be the case.

Each emission line included in the blend is entered on its own line. This list
begins on the line after the \cdCommand{set blend} command and continues until
a line with \cdCommand{end} in the first three columns appears. The format for
entering the spectral lines is described in
Section~\ref{sec:SpecifySpectralLines}. An arbitrary number of lines can be
entered into the blend, and there is also no limit to the number of blends
that can be defined in this way. An example is shown below.
\begin{verbatim}
set blend 3968
Ne 3  3967.47
H  1  3970.07
end
\end{verbatim}

The blend components need to have type 't' as defined in the \cdCommand{save
  line labels} output and also be associated to a transition. The latter can
be recognized from the fact that the comment in the \cdCommand{save line
  labels} output has entries for ``index'' and ``Elow''. The custom blends
will be stored on the line stack and can be used like any other entry on that
stack.

If the keyword \cdCommand{quiet} appears on the \cdCommand{set blend} command
line, and the database model for any of the species in the blend has been
turned off, the blend will be quietly skipped rather than producing error
messages about missing lines. This is useful if the \cdCommand{set blend}
command is included in an init file that will be shared between many scripts.
However, note that the code cannot discriminate between a species that has
been disabled and a typo in the label, so make sure that the labels are
correctly typed before using this keyword.

An init file called \cdFilename{blends.ini} with commonly used multiplets and
blends is included in the data directory\footnote{For \Cloudy\ versions C17 and
  older these blends were hardwired into the source code.}. \Cloudy\ automatically
parses this file unless you include the \cdCommand{no blends} command in your
script. It is possible to modify this file to add your own blends (but note that
this file is used by the test suite, so modifying or deleting existing blends
could affect subsequent test suite runs). You can also use this file as a
template to create your own init file with custom blends, or you can enter the
blends directly in your script.

\subsection{Set charge transfer -11.5}

This command establishes the coefficient in the statistical hydrogen
charge transfer rate coefficients used for species more than four times
ionized (\citealp{Ferland1997}).
If the number is negative then it is assumed
to be the log of the coefficient,
if zero then this estimate is turned off,
and if positive the number is the coefficient itself.
It is stored as the
variable \cdVariable{HCTMin} and has the default of
$1.92\times 10^{-9}$ cm$^3$ s$^{-1}$.
This is used to
set a rate coefficient of \cdVariable{HCTMin} \%$q$
where $q$ is the excess charge of the heavy element.

\subsection{Set chemistry}

This changes some aspects of the chemistry network.
\begin{description}
\item[set chemistry Federman  [ON , OFF]]  This determines whether the rate
coefficients derived by \citet{Zsargo2003} are used.
Their
chemical network consists of a small subset of reactions important to the
formation of CH, CO, HCO$^+$, CH$_{2^{+}}$, and CH$_{3{^+}}$ in regions with large amounts
of CH$^+$.
The default is ``\cdCommand{ON}'', that is, to use these rates instead of UMIST
in the reaction network.
If \cdCommand{OFF} is specified then UMIST rates are used.

\item[set chemistry non equilibrium] This uses the chemical model of \citet{Federman1996} to examine the effects of nonthermal chemistry on diffuse
cloud chemistry.
This command uses an effective kinetic temperature for
a chemical reaction that is caused by ion / molecule slip
when turbulence in present.
The effective temperature is given by
\begin{equation}
T_{eff}  = \frac{3}{2}kT + \frac{1}{2}\mu \Delta u_{turb}^2
\end{equation}
where $k$ is Boltzmann's constant, $T$ is the gas kinetic temperature, $\mu$ is
the reduced mass of the two reactants, and $\Delta u_{turb}$ is the turbulent velocity.
\citet{Zsargo2003} summarize the physical mechanism of
non-thermal chemistry as:
\begin{quote}
``The origin of the nonthermal motion is assumed to be the result of
the propagation of Alfv\'en waves, constantly entering the cloud from the
intercloud medium.
The dissipation of these waves is concentrated primarily
in a diffuse cloud boundary layer,
the transition zone between the cloud
and the intercloud medium.''
\end{quote}
   The effect of $T_{eff}$ is to increase the rates of those reactions which
have a strong exp($-c/T$) temperature barrier.
This command primarily effects
the formation of CH$^+$ (the abundance of which cannot be explained by
equilibrium chemistry models) by eliminating the temperature barrier for
the C$^+$ + \htwo\ $\to$ CH$^+$ + H formation channel, allowing CH$^+$ to form.
The formation of CH$^+$ has trickle down effects on the predicted abundances
of other molecules,
such as CH, C$_2$, CN, and OH (see equations 7 \& 8 of
\citealp{Zsargo2003}).
This option was added by Nick Abel.
The keyword can
be \cdCommand{non equilibrium} or \cdCommand{non-equilibrium}.

\item[set chemistry non equilibrium neutrals [ON , OFF]]    If the option
\cdCommand{neutrals off} is also included then $T_{eff}$ will equal $T$ for all neutral-neutral
reactions, so non-equilibrium chemistry is not included for these species.
The physical assumption is that neutrals are not coupled to the magnetic
field and therefore do not experience the effects of nonthermal chemistry.
This was used by \citet{Federman1996} to explain low OH abundances in
regions with high CH$^+$, such as $\xi$ Per.
By default, neutral-neutral reactions
are included in the non-equilibrium chemistry.

\item[set chemistry proton elimination  [ON , OFF]]  In ``proton elimination''
reactions, an atom-molecule non-charge transfer reaction leads to the
formation of either H or \hplus.
An example is C$^+$ + OH which leads to either
CO + \hplus\ or CO$^+$ + H.  \citet{Huntress1977} states that the energetics of the
reaction makes the \hplus\ formation channel highly unlikely.
This command sets
the rates of these reactions leading to \hplus\ (which currently only includes
the example) to zero.
By default, proton elimination is included.
\end{description}

\subsection{Set check energy every zone}
The code checks that it has not used more energy than
is available in the incident radiation field when the calculation ends.
This command will cause that check to occur for every zone. 
This does slow down the calculation somewhat, so it a debugging
tool to use when energy conservation problems occur.

\subsection{Set collisional ionization [Dima, Hybrid]}
The collisional ionization rates of \citet{Voronov1997},
as implemented by Dima Verner soon after publication of that paper, 
were used through version C10.
These are complete over the range of temperatures considered by \Cloudy .
\citet{Dere.K07Ionization-rate-coefficients-for-the-elements} 
presents updated rates but these do not cover the temperature range we need.
The \citet{Voronov1997} and \citet{Dere.K07Ionization-rate-coefficients-for-the-elements} 
data are generally in excellent agreement.
We scale the \citet{Voronov1997} rates by the ratio of the 
\citet{Dere.K07Ionization-rate-coefficients-for-the-elements} to \citet{Voronov1997}
to obtain updated rates over the full temperature range.  
We refer to this as the hybrid scheme.
The scaling factor for each species is the Dere coefficient from the 
center of the temperature range given in 
\citet{Dere.K07Ionization-rate-coefficients-for-the-elements} 
divided by the 
\citet{Voronov1997} coefficient at that same temperature.
The \cdCommand{set collisional ionization Dima} uses the \citet{Voronov1997} rates
while \cdCommand{set collisional ionization Hybrid}, the default, uses thsi scaling.

\subsection{Set collision strength averaging on}

This controls whether the $l$-mixing collision strengths
used for hydrogenic
levels are evaluated at $kT$ (the default) or are an average over
the thermal Maxwellian.
If the keyword \cdCommand{on} appears then averaging will be
done---this
is very slow due to the large number of levels and evaluations involved.
Tests show that the default, the evaluation at a single energy of $kT$, is
significantly faster and produces very similar predictions.

\subsection{Set continuum options }
\label{sec:CommandSetContinuumOptions}
\begin{description}
\item[set continuum resolution 0.1]  This changes the resolution of the coarse
continuum energy mesh by a constant scale factor.  The resolution of the
continuum mesh as a function of energy is defined by the data file
\cdFilename{continuum\_mesh.ini}, which is located in the data directory.  Permanent
changes to the continuum resolution should be made there.
This command
allows the continuum mesh to be changed by a temporary scale factor.
The
number on the command line multiplies the resolution used by the code.
Factors less then unity make the cells smaller, for higher
resolving power, while factors greater than unity make the resolution coarser.
For instance, an entry of 0.1 would make the resolution ten times finer
or the resolving power ten times greater.
If the number is less than or equal to zero
it is interpreted as the log of the resolution scale factor.

This was originally added so that the entire continuum resolution
could be improved by large factors when running some of the
fundamental tests
of hydrogenic and helium-like emission.
The code's execution speed is
affected by the number of coarse continuum cells due to the frequent
reevaluations of the opacity and photo-interaction rates.
For factors near
unity the execution time will scale roughly inversely linearly.

The code can be sped up by making the factor larger than unity.
But
there is a limit where the computed results will be strongly affected
and the code may even no longer pass its internal sanity checks.
A factor
near 2 will cut the execution time by about 40\%.
This will affect the
predictions and should only be done in exploratory calculations.

This will change the line to continuum contrast since emission lines are not resolved in the coarse continuum.
See the discussion around Equation \ref{eqn:LineContinuumContrastFactor} for more details.

\item[set continuum shielding] This changes the treatment of shielding
  of the incident continuum by line optical depths.  There are several
  options.  \cdCommand{Federman} is the default and uses the function
  given in the appendix of \citet{Federman1979} with a bug fixed in
  the wing shielding; \cdCommand{Fbug} uses the original function from
  \citet{Federman1979} without the bug fix, as used by versions of
  \Cloudy\ up to 13.02; \cdCommand{Ferland} uses the function given by
  \citet{Ferland1992}; \cdCommand{Rodgers} uses a function derived
  from \citet{Rodgers1974}, as used by \citet{Draine1996}.
  \cdCommand{Integral} uses an explicit numerical integral over the
  Voigt function, which should be highly accurate but is also very
  slow. \cdCommand{Pesc} uses the inward-looking escape probability
  (though note that the escape probability includes an integral over
  which is not needed in this case). \cdCommand{None} turns off
  self-shielding.

\item[set FeII continuum]
The command is obsolete.
See \cdCommand{set species continuum}, Section~\ref{sec:SetSpeciesContinuum}.

\label{sec:CommandSetFineContinuum}
\item[set fine continuum]  This changes the logic used
to determine the
resolution of the fine continuum mesh.
The command has two arguments.
The name of an element must appear on the line---the a
tomic weight of this
element will be used to determine the narrowest lines
that must be resolved.
The number of resolution bins over a HWHM must also
appear on the line.

\begin{shaded}
A \cdCommand{range} option can be used with the
\cdCommand{set fine continuum} command.  The first number is
interpreted as the lower limit (in Rydbergs) of the fine continuum.
An optional second number modifies the default upper limit.  For both
parameters, negative numbers are interpreted as logs.
This command is experimental
and can appreciably change results in conditions where line overlap
is important.
\end{shaded}

\end{description}

\subsection{Set convergence \dots}

The error in the heating-cooling balance is set
with the \cdCommand{set temperature
convergence} command.
The error in the electron
density is set with the \cdCommand{set eden convergence} command.
The error in the local pressure is set with the
\cdCommand{set pressure convergence} command.
In cases where the code
is having trouble converging the \cdCommand{set presioniz} command
will limit the number of ionization evaluations before declaring
a failure.

\subsection{Set coverage [fast]}

This limits the number of zones and iterations that will be done so that
the code runs more quickly.
It allows quick runs of the test suite with
tools such as \cdRoutine{Purify} and \cdRoutine{valgrind}.
If the \cdCommand{fast} option appears then only
one and one iteration will be performed.
Otherwise up to ten zones and
two iterations will be done.
This limit is imposed on top of whatever other
limits may exist in the input stream.

\subsection{Set csupra = -12.34}

This sets the \hO\ secondary ionization rate due to supra-thermal electrons
to the number on the line.
The number is the log of the rate [s$^{-1}$].
The
excitation rate of L$\alpha $ is assumed to be the same.
This option is used to
test the code in secondary-ionization dominated cases.
Normally the
secondary ionization rate is determined self-consistently from the
high-energy radiation field or the cosmic ray density.

\subsection{Set cumulative [ mass, flux, off ]}
\label{sec:SetCumulativeCommand}

Set the type of accumulation to be used for \cdCommand{save
  cumulative} output.  Options are 
\begin{itemize}
\item \cdCommand{mass}, which is
appropriate for single-element cooling calculations and generates
output in $\rm erg\,g^{-1}$, 
\item \cdCommand{flux}, which is
appropriate for calculations on fixed domains and generates
output in $\rm erg\,cm^{-2}$, and
\item \cdCommand{off} which disables the cumulative calculation, and
will generate an error if a 'save cumulative' output is encountered.
\end{itemize}

\subsection{Set D/H -3.2 -- DEPRECATED}

This command is deprecated.
Use \cdCommand{element hydrogen isotopes} instead.

\subsection{Set density tolerance}

Set the maximum relative error in the conservation of
the gas phase density for each element, permitted
by the ionization and molecular balance solvers.

\subsection{Set didz 0.05 }

The thickness of the first zone is chosen so that the largest continuous
optical depth through it is one percent of the entered value.
Thereafter
the zone thickness is continuously adjusted by checking that the optical
depth at the maximum continuum-gas interaction energy is set to this value.
The default is 0.15.
If the value is less than or equal to zero, then it
is interpreted as the log of the quantity, and linear if greater than zero.

\subsection{Set dielectronic recombination [keywords]}
\label{sec:SetRecombination}

This changes
the treatment of various details of the ionic recombination physics.
The \cdCommand{print recombination} command described on page
\pageref{sec:PrintRecombination} prints a summary of the data in use.

\begin{description}
\item[set dielectronic recombination mean [options]]
At present rate
coefficients for dielectronic recombination through low-lying autoionizing
states have not been computed for lower stages of ionization of most
elements on the third row (of the periodic table) and higher.
The \cdCommand{mean} option modifies the treatment
of the estimates of the missing dielectronic recombination coefficients.

The default is to use means of all existing DR rates for each ion stage
to replace missing data for that stage.
For instance, Fe$^+ \Rightarrow \mathrm{Fe}^0$ does not
have reliable DR rates.
Its rate will be the mean DR rate for all first
ions of all elements with published rates.

\item[set dielectronic recombination mean off]
will turn off guesses for DR rates.
These guesses are normally used for those ions that
have no rate coefficients because no better can be done at present.

\item[set dielectronic recombination mean scale]
will multiply the DR guesses by scale factors.
One or more scale factors may appear on the line, and each will
multiply the mean rate for successive ions.
If there are fewer scale factors than ions then the last
scale factor entered will be used for all higher ions.
The scale factors must be greater than or equal to zero.

\item[set dielectronic recombination mean noise [dispersion]]
will introduce Gaussian noise with the indicated dispersion.
This provides
a way to quantify the uncertainty caused by the need to use mean rates.
This adds a component of Gaussian random noise to the means of the
dielectronic recombination rate coefficients.
The number on the line is the dispersion.
The resulting random number will multiply the rate coefficient.
As a result the same DR rates will be used throughout
any single calculation but the noise multiplying the rate will change from
run to run.

\item[set dielectronic recombination suppression off]
will turn off suppression of dielectronic recombination.
The DR rate depends on both the gas temperature and density,
such that as the density increases the DR rate decreases.
This is parametrized as a suppression factor relative to the
zero-density value, according to \citet{Badnell2013-DRsuppression}.
Unless this command is issued, the suppression will be applied.
\end{description}

\subsection{Set dr 11.2}

This forces the zone thickness to a prescribed value.
A space is required after the ``\cdCommand{dr\_}''.
The argument is interpreted as a logarithm by default.
The argument is interpreted as the linear value if the
keyword \cdCommand{linear} appears.
If the keyword \cdCommand{relative} is present the
value will be interpreted as a relative fraction of the current radius. In that
case the value needs to be between 0 and 1. The default is that this command
is not in effect and the adaptive stepsize algorithm will set the zone thickness
in accordance with the local physical conditions.

Use this command with great caution! Forcing the code to take steps that are too large
can lead to inaccurate results, or even destabilize the code.

\subsection{Set drmax 11.2}

This sets the largest allowed zone thickness [cm].
The zone thickness
will not become larger than this.
The argument is a logarithm
if it is less than 38 or if the keyword \cdCommand{log} appears.
It is linear if the keyword
\cdCommand{linear} appears. If the keyword \cdCommand{relative} is present the
value will be interpreted as a relative fraction of the current radius. In that
case the value needs to be between 0 and 1.
The default for the maximum zone thickness is $10^{30}$~cm.

\subsection{Set drmin 11.2}

This sets the minimum zone thickness [cm]. The zone thickness will not become
smaller than this, except possibly in the last zone to avoid overshooting the
outer radius. The argument is a logarithm if it is less than 38 or
if the keyword \cdCommand{log} appears. It is linear if the keyword
\cdCommand{linear} appears. If the keyword \cdCommand{relative} is present the
value will be interpreted as a relative fraction of the current radius. In that
case the value needs to be between 0 and 1.

The keyword \cdCommand{depth} works similarly to the keyword \cdCommand{relative}
but sets the minimum zone thickness relative to the depth into the cloud.  
There is a default value of $10^{-5}$.  This minimum is not active if a
\cdCommand{set dr} command is used. 

The default for the minimum zone thickness is the greater of approximately 
$1.2 \times 10^{-36}$~cm and the default relative to depth.

Use this command with great caution! Forcing the code to take steps that are too large
can lead to inaccurate results, or even destabilize the code.

\subsection{Set dynamics commands}

These commands change aspects of dynamical or time-dependent calculations.
They are described in 
\ref{Hazy1-sec:DynamicalTimeDependent}
\cdSectionTitle{\refname{Hazy1-sec:DynamicalTimeDependent}} 

\subsection{Set eden [options ....]}
\label{sec:CommandSetEden}
These commands provide several ways to set the electron density
or change its convergence criterion.

\begin{description}

\item[set eden 2.3]   If no other keywords are recognized then this command
sets the electron density.
The number is the log of the density [$\pcc$].

\item[set eden convergence 0.01]
This sets the convergence criterion for the electron density.
The number is the largest relative error in the electron density.
It is interpreted as the log of the relative error if it is negative.
The default value is 0.01.
The old form of the command,
\cdCommand{set eden error}, is also recognized.

\item[set eden fraction 1e-3]
This sets the electron fraction, $n_e / n(\mathrm{H})$,
where $n(\mathrm{H})$ is the density of hydrogen in all forms.
The number is the electron fraction and is interpreted as a log if it is is $\leq 0$.

\item[set eden solver [ vWDB , SECAnt {]} ]
Switch between vWDB electron density solver (the default) and experimental
limited secant solver.

\end{description}

Note:  The \cdCommand{eden} command,
described on page \pageref{sec:CommandEden}, add an extra source
of electrons.
These \cdCommand{set eden} commands set the electron density itself.
These commands violate charge conservation.

\subsection{Set flxfnt -20}

The highest photon energy that must be considered is lower for relatively
soft continua such as a star than for X-ray sources such as AGNs.
The
criterion used to choose the highest energy to be considered
$\nu_{high}$ is that
$\nu f_\nu (\nu_{high}) /\nu f_\nu (\nu_{peak}) < $ \cdTerm{flxfnt},
where $\nu_{peak}$ is the frequency where
the continuum reaches its maximum $\nu f_\nu$.
\cdTerm{FluxFaint} is normally $10^{-10}$.
This
command changes the value of \cdTerm{FluxFaint}.
The argument is the log of the value.

\subsection{Set gbar off}
\label{sec:Setgbar}
By default the g-bar approximation is used for transitions in the 
Chianti and Stout databases 
which have no collision strengths.
G-bar values are taken from \citet{Mewe1972} for both allowed and forbidden transitions.
For simplicity, transitions with gf values greater than or equal to $10^{-8}$ are considered to be allowed.

This command says not to use g-bar for any database transition. 
If there is no collision data and gbar is turned off , 
then $0.01$ is used for the collision strength.

\subsection{Set grain heating factor 2.}

This command allows you to enter a fudge factor to alter the amount of
photoelectric heating from the grains (also when the \cdCommand{set PAH Bakes}
command is in effect). The number you enter is a multiplicative factor and
is always interpreted as a linear number. This command violates energy
conservation and is intended for testing only.

\subsection{Set H2 \dots}
\label{sec:SetH2}

This command changes some aspects of the \htwo\ molecule.
Other aspects
of this molecule are changed with the \cdCommand{database H2} command
described on page \pageref{sec:AtomH2} above.

\begin{description}
\item[set H2 fraction]  This sets the ratio $n$(\htwo)/$n$(H).
The number is
interpreted as the log of the ratio if it is less then or equal to zero
and as the linear ratio otherwise.
An upper limit of 0.5 will be silently
imposed.
This command turns off the normal \htwo\ molecular network and forces
the H2 density to the hydrogen density multiplied by this fraction.
This
is totally unphysical and is only intended to provide an easy way to test
the \htwo\ molecule.

\item[set H2 formation scale]  This scales the H2 formation rate
(via three-body recombination). The number is
interpreted as the log of the scaling factor if it is less then or equal to zero
and as the linear factor otherwise.  In the fully molecular or fully
ionized limits, this command should be used rather than the
\cdCommand{set H2 fraction} command because it allows the same functionality without
breaking numerous consistency checks.

\item[set H2 Jura [TH85; CT02; SN99; ELRD]]  This changes the ``Jura rate'', the rate
\htwo\ forms on grains \citep{Jura1975}.
The default  rates  include the
Eley-Rideal Effect \citep{LeBourlot12, Rollig13} as implemented by
\citet{Rollig13}.
These depend on grain temperature and abundance.
The option \cdCommand{TH85} will use rates from \citet{Tielens1985a},  \cdCommand{SN99} 
will use rates for \citet{Sternberg1999},  \cdCommand{CT02} uses \citet{Cazaux2002}.
Finally, \cdCommand{ELRD} will use the rates that include the
Eley-Rideal Effect \citep{LeBourlot12, Rollig13} and is the default.

\item[set H2 Jura scale]   The number is a linear scale factor that
multiplies the formation rate obtained by one of the above methods.  If
the scale factor is negative or the keyword \cdCommand{log} appears then the number
is interpreted as the log of the scale factor.
\citet{Habart2004} suggest
that the rate can vary across the galaxy by large factors.

\item[set H2 Jura rate]  The number is the log of the \htwo\ grain
formation rate $k$ [cm$^3$~s$^{-1}$].  This is used to give a simple grain formation
rate $R = k\,n\left( {{\mathrm{H}}_{tot} } \right)n\left( {\hO } \right)$
 [cm$^{-3}$~s$^{-1}$].  If no number appears on the line then the \citet{Jura1975} rate
of $3 \times 10^{-17}$ will be assumed.

\item[set H2 Solomon [BD96; TH85; ELWERT]]  This changes how the Solomon process
is treated when the small \htwo\ molecule is used.  The fits to the dissociation
rate derived by \citet{Elwert2006} are the default and will be used if
the keyword \cdCommand{ELWERT} appears.  The keyword \cdCommand{BD96} will use \citet{Bertoldi1996} and the keyword \cdCommand{TH85} will use the \citet{Tielens1985a} rate.
These rates differ by $\ge1$ dex for some \htwo\ column densities.  When the large
\htwo\ molecule is used, as set with the \cdCommand{database H2} command,
the Solomon
process is treated in a self-consistent manner.
\citet{Abel2004} show
that \htwo\ column densities are quite sensitive to the correct
treatment of this physics for translucent clouds.

\item[set H2 continuum dissociation [Stancil; AD69]]  
This command changes how direct photodissociation
of \htwo\ is treated.  
By default, \Cloudy\ computes the photodissociation rate 
by integrating the cross-section
for photodissociation over the energy range for dissociation.  
The cross sections used for this process come from 
Stancil et al. (in preparation) and are specified by the 
keyword \cdCommand{Stancil}.  
The other option 
is to use a constant \htwo\ cross section for this process, 
which comes for the work of \citet{Allison1969} and 
is used when \cdCommand{AD69} appears on the command line.  
This process, while not the dominant destruction
process for \htwo\ in a PDR, may be important in some environments.

\item[set H2 grain formation pumping]  This changes
the distribution function
for populations of $v$, $J$ levels of \htwo\ following formation on grain surfaces.
There are several possible
distributions: \citet[\cdCommand{Taka}, this is the default]{Takahashi2001};
\citet[\cdCommand{DB96}]{Draine1996}; and thermal (\cdCommand{thermal}).

The keyword \cdCommand{ OFF} will put all newly formed \htwo\ in the lowest ortho and
para states with a 3:1 ratio.
This can be used to test the importantence of grain formation pumping.
For low to moderate temperatures in a purely collisional gas
the higher levels within X may have no 
source of populations when this is done.
The code may then fail because the level population matrix will become ill conditioned.
This option should only be used for experiments.

\item[set H2 Tad 420]  This changes the binding energy,
$T_{ad}$, in the formalism
described by \citet{LeBourlot2000} to account for deexcitation
of \htwo\ while on
a grain surface.
The number is the binding energy expressed as a
temperature.
The default value of 800 K maximizes the rates of \htwo\ collisional deexcitation and ortho-para conversion.
If the number is
$\le 10$ it is interpreted as a log of the temperature
unless the keyword \cdCommand{linear} appears.
\end{description}

\subsection{Set HCORR}

The density used in electronic collisions is actually $n_e  + 1.7 \times
10^{4} n\left( {\hO } \right)f$.
The second term is an approximate correction for collisions by neutral
hydrogen.
This is only a guestamite based on \citet{Drawin1969},
(see also \citealp{Weisheit1974},
\citealp{Steenbock1984}, \citealp{Lambert1993}, and \citealp{Kiselman2000}).
The
number on this command adjusts the scale factor~$f$.
You can turn off the
process by setting the number to zero.

\subsection{Set HO charge transfer [chemistry; ionization]}

This command is no longer supported.

\subsection{Set ind2 [on, off]}

This provides a way to turn off the affects of induced two-photon emission
or absorption.
One of the keywords \cdCommand{on} or \cdCommand{off} must appear
to turn induced
two-photon processes on or off.
It is on by default.

\subsection{Set ionization tolerance 0.01}

Sets the allowed error tolerance on ionization solver solutions.

\subsection{Set isotopes [ALL] -- DEPRECATED}

This command has been deprecated.
Use \cdCommand{element isotopes all} instead.

\subsection{Set kshell energy 1e6}

This changes the energy of the highest continuum point considered for
photoelectric opacity.
The default is 1 MeV, sufficiently high that Compton
recoil and pair production are the dominant opacity sources and K-shell
opacity may safely be ignored.
Setting this limit to smaller values will
save some compute time since the evaluation of the photoionization rate
integrals will not extend to as high an energy.
The argument is the energy
in Rydbergs and it must be greater than 194 Ryd.
If zero is entered then
the high-energy limit of the continuum will be used.

\subsection{Set Leiden hack}

This replaces certain physical processes with the simple prescriptions
that were specified for the 2004 Leiden PDR workshop
(\citealp{Roellig2007}).
These are only intended for testing and would not be used
in a real calculation.
If any \cdCommand{set Leiden hack} commands are entered the code will
complain that a physical process has been disabled.

This enables all hacks.
Individual hacks can be turned on with additional keywords.
They are:

\begin{description}
\item[set Leiden hack H2* off]  The chemistry network used in the workshop did
not include \htwo* (see \citealp{Tielens1985a} for a definition).  This
turns off reactions between that species and the CO network (although it
is still present in the H-\htwo\ network).

\item[set Leiden hack CR off]  This turns off cosmic-ray excitation for the
large \htwo\ molecule.
This is normally treated as in \citet{Dalgarno1999}.

\item[set Leiden hack UMIST]  This forces the code to use parts of the UMIST
database that (we feel) we do better with other methods.
\end{description}

\subsection{set monitor scientific notation}
\cdCommand{monitors} are commands that
tell the code what value for a predicted quantity to expect.  These are
part of the test suite and allow automatic testing of the code's results.
This command causes the printed monitored values to be printed in scientific
notation rather than a float format.
The \cdCommand{monitor} command is described on page
\pageref{sec:CommandMonitor} above.

\subsection{Set nchrg 3}

The grain physics uses an \emph{n}-charge state model where, for each bin,
the charge distribution is resolved in exactly $n$ charge states,
independent of grain size.
This is discussed in \citet{VanHoof2004}.
The default
value of $n$ is 2 but can be reset with this command
to any value between 2 and 10.
Higher values of $n$ will give more accurate results at the expense
of greater compute time.

\subsection{Set negopc}

Negative opacities may occur during a calculation if a level happens
to mase (\citealp{Ferland1993}).
The code will generate a comment at the end if
this happens.
This command tells the code to save the optical depth array
when negative opacities occur.
The output will go to the file
\cdFilename{negopc.txt}.

\subsection{Set nend 500}

This sets the default limit to the number of zones that will be computed.
The preset default limiting number of zones is 1400 but more zones may be
needed in large column density clouds or ones exposed to very intense
radiation fields.

The limit to the number of zones that will be computed can be set with
either this command or with the s\cdCommand{top zone} command.
The only difference
is in the level of warning that will be generated if the code stops after
reaching the limiting number of zones.
If the code stops because it reached
the number of zones set by the \cdCommand{stop zone} command
then it thinks that this
was the intended stopping criterion and no comment is generated.
However
the code generates a \cdTerm{warning} if it stops because
it reaches the default
limit to the number of zones since this probably \emph{was not} intended.  By using
this command the limit can be increased while still retaining the checking
and warnings that are generated if the code stops for an unintended reason.

The code allocates memory to store a great deal of information for each
zone.
Increasing the number of zones will also increase the memory needed
to run the code.

\subsection{Set nFnu [options]}
\label{sec:set:nfnu}

By default, \Cloudy\ will print
the continuum flux at certain wavelengths as a series of entries in the
standard emission line array.
The \cdCommand{set nFnu} command changes which parts of the
radiation field  are included in the entry
with the \emph{nFnu} label.
There are four possible radiation fields, the transmitted
and reflected incident radiation field,
and the transmitted and reflected diffuse
emission produced by the nebula.
The default is to include only the 
reflected incident radiation field
and the transmitted and reflected diffuse
emission.  The transmitted incident is not included by default.
Any or all of these may be entered on the command line, but at least one must occur.
Any component left off the command will not be included.

The four keywords, and the component that is included, are the following:

\cdCommand{incident reflected}  or \cdCommand{incident\_reflected} includes
the incident reflected continuum.

\cdCommand{incident transmitted}  or \cdCommand{incident\_transmitted} includes
the incident transmitted continuum.

\cdCommand{diffuse inward}  or \cdCommand{diffuse\_inward} includes the diffuse
inward continuum.

\cdCommand{diffuse outward}  or \cdCommand{diffuse\_outward} includes the
diffuse outward continuum.

\cdCommand{total}  includes all four components of the continuum.

All desired components must be specified on a single \cdCommand{set nFnu} command.
The following
shows how to only include the outward incident and outward diffuse emission
in the \emph{nFnu} entry in the printout:
\begin{verbatim}
set nFnu diffuse outward, incident transmitted
\end{verbatim}

This command
does {\em not} change the quantities printed with the \emph{nInu} label---these
are always the sum of the reflected plus transmitted incident continuum.

\subsection{Set nFnu add 300 micron}
\label{sec:set:nfnu:add}

You can also add custom wavelengths or frequencies at which the \emph{nFnu}
and \emph{nInu} entries in the emission-line printout are evaluated. You can
do that by adding one \cdCommand{set nFnu add} command for each point. You can
enter as many as you like. Each line must contain a number and a unit to identify
the wavelength or frequency. Adding the unit is mandatory. The following keywords
are supported: \cdCommand{eV}, \cdCommand{keV}, \cdCommand{MeV},
\cdCommand{Hz}, \cdCommand{kHz}, \cdCommand{MHz}, \cdCommand{GHz},
\cdCommand{Angstrom}, \cdCommand{nm}, \cdCommand{micron}, \cdCommand{mm},
\cdCommand{cm} or \cdCommand{centimeter}, 
\cdCommand{wavenumbers}, \cdCommand{erg}, \cdCommand{ryd}, \cdCommand{K} or
\cdCommand{Kelvin}.
Some examples are:
\begin{verbatim}
set nFnu add 22 GHz
set nFnu add 350 micron
set nFnu add 4200 angstrom
\end{verbatim}

\subsection{Set nmaps 50}

This controls the number of steps in the heating-cooling map that results
from either the \cdCommand{map} or \cdCommand{save map} commands.  Normally about 20 steps are
taken between the lowest and highest temperatures.

\subsection{Set numerical derivatives}

This tells the code to use numerical rather than analytic derivatives
of the heating and cooling functions.
The default is to use the analytic derivatives.

\subsection{Set PAH option}
\label{sec:CommandSetPahOption}

The \cdCommand{grains PAH} command described on 
page \pageref{sec:GrainPAHcommands} adds PAHs to the calculation.
These \cdCommand{set} commands change some details of their treatment.

\begin{description}
\item[set PAH constant or ``H'']
This changes the function that describes how
PAH abundances depend on physical conditions.  The default, which can also
be specified with the string \verb|"H"|, is for the PAH abundance to
scale as $n$(\hO ) / $n$(H$_{tot}$).  You can also specify the PAH abundance
to scale as the total abundance of neutral atomic and molecular hydrogen
with the string \verb|"H,H2"|.
The abundance will be constant if the keyword
\cdCommand{constant} appears.

\item[set PAH Bakes] will replace our self-consistent
treatment of photoelectric heating by all grains
with the PAH photoelectric heating equation given by \citet{Bakes1994}.
This command no longer guarantees energy conservation and is only intended for testing.
\end{description}

\subsection{Set phfit [1995, 1996]}

The key \cdCommand{1995} tells the code to use photoionization
cross sections from \citet{Verner1995}.
The key \cdCommand{1996} is the default and uses
\citet{VernerFerlandKorista1996}, which is partially based
on Opacity Project cross sections.

\subsection{Set pressure convergence 0.01}

This sets the convergence criterion for the total pressure.
The number
is the largest relative error in the pressure.
The number is interpreted
as the linear error if it is positive and the log of this error if negative.
The default value is 0.01.

\subsection{Set pressure ionize 50}

This sets a limit to the number of times the bottom ionization solver
will be called from the top pressure solver within a zone.
This is a debugging aid as well as a means to limit the time it
takes the code to declare convergence failure.
The limit has a default value of 3000.
This is approximately 500 more calls than is ever needed in any test suite simulations.
The vast majority ($\sim99\%$) of zones are converged in less than 100 calls.
Users should exercise caution in increasing the limit.
Encountering the default limit is a strong indication that something
is wrong with the simulation.
It is unlikely the simulation will converge if more calls are allowed.

\subsection{Set save commands}

Save output is the primary output mechanism for the code.
This command sets various details about
how this output is generated.

\begin{description}
\item[set save flush] will cause the file buffers to be flushed after every iteration.

\item[set save hash ``newstring'']
\label{sec:CommandSetSaveHash}
When more than one iteration is saved
into a file each ends with a series of hash marks that makes it easy to
locate with an editor.
This sets the hash string to something else.  A
new string must appear within the pair of double quotes.

The special string ``\cdCommand{return}'' is replaced
with a single carriage return
which translates into the pair of carriage returns that \cdTerm{gnuplot} uses to
separate plots.
The special string ``\cdCommand{time}'' will give the elapsed time
since the start of time-dependent calculations, followed by a pair
of carriage returns.
In both cases the keyword
must appear within the pair of double quotes.

The hash can be completely turned off with the \cdCommand{no hash}
option on the \cdCommand{save} command.

\item[set save line width / resolving power, suppress, absorption]
\label{sec:CommandSetSaveLWidth}
The observed contrast between emission
lines and the continuum depends on the intrinsic line width and,
for an
unresolved line, the resolution of the spectrometer.
Lines are included
in the radiation field produced by the \cdCommand{save continuum}
commands. The default behavior is to do this in such a way that
the energy is conserved in the resulting output.\footnote{In C90.00 the
standard was to use a line width equal to the speed of light. In C90.04 the
standard was changed to using a line width of $1000 \kmps$. Finally, in
version C10 the standard was changed again to using energy conservation.
Versions prior to C90.00 did not include emission lines in the saved
spectral energy distribution.}
This command allows you to adjust the contrast between the lines and continuum by,
in effect, changing the velocity resolution of the spectrometer which views
the unresolved lines.

Lines and continua are stored separately throughout the code.
They are
combined only when the output from the \cdCommand{save continuum} command is produced using the expression
\begin{equation}
\nu F_\nu  ({\mathrm{total}}) = \nu F_\nu  ({\mathrm{continuum}}) +
R \, I\left( {{\mathrm{line}}} \right)
\label{eqn:LineContinuumContrastFactor}
\end{equation}
where $R \equiv \lambda/\Delta\lambda$ is the resolving power and $\Delta\lambda$ 
is the resolution. 

The
default behavior is to match $R$ to the resolving power of the coarse continuum
at the frequency of the line as set in the
the file \cdFilename{continuum\_mesh.ini}. This assures energy
conservation. If you want to boost weak lines in a plot, you can artificially
change the resolution of the spectrum by using this command.

When you use the \cdCommand{set save line width} command, you should supply a
velocity width $\Delta v$ in km s$^{-1}$. This will set the resolving power $R = c/\Delta v$,
where $c$ is the speed of light. If you use the alternative 
\cdCommand{set  save resolving power} command, you can specify $R$ directly. 
Supplying a smaller
value of $\Delta v$ or a higher value of $R$ will increase the line-to-continuum
ratio and thereby make lines appear stronger.
\begin{verbatim}
## set velocity resolution to 100 km/s => R = 3000 (approximately)
set save line width 100
## alternatively set spectral resolving power directly
set save resolving power 3000
\end{verbatim}

Note that this command will {\em only} adjust the height of the line, not the
width. The latter will always be the width of one cell in the coarse continuum mesh
(even if the line is broader than that cell). This implies that
the line flux in the continuum array is artificially changed by this command.
Changing the spectral resolution can be useful to emphasize weak lines in a plot,
but should never be used when energy conservation is important, e.g. when
you want to fold the saved continuum with a photometric passband.

If the keyword \cdCommand{absorption} occurs then the effects
of absorption lines on the transmission of the incident continuum
will be modified by the factor given in
equation \ref{eqn:AbsLineContinuumContrastFactor}.
The transmitted radiation field is changed as
\begin{equation}
\nu f_\nu   = \nu f_\nu ^0 \max \left[ {1 - \left( {1 - trans} \right)R\frac{\Delta\nu}{\nu}} , 0 \right]
\label{eqn:AbsLineContinuumContrastFactor}
\end{equation}
where \cdVariable{trans} is the correct transmission factor as propagated
from the fine to coarse mesh, and $\nu/\Delta\nu$ is the resolution of the coarse continuum mesh.
The keyword \cdCommand{absorption} makes the effects of absorption appear larger than than
actually are, but gives an impression their appearance would have if
the spectrum were viewed at a higher resolution. This approach
uses the opacities stored in the much higher resolution fine continuum.
As is the case for emission lines, using this option will violate energy conservation.

If you include the keyword \cdCommand{suppress} on the command line, then
emission lines will not be included in the saved spectral energy distribution.
If you combine the keywords \cdCommand{suppress} and \cdCommand{absorption}, then
the effects of absorption lines will also be suppressed and only a pure continuum
will be saved.

The only effect of this command is to change the line-to-continuum
contrast in output from all \cdCommand{save} commands
except for that produced by the
\cdCommand{save transmitted continuum} command.

\item[set save luminosity old]  
\label{sec:SetSaveLuminosity}
In versions of Cloudy between
roughly 1983 and C13 the \cdCommand{save continuum} output
reported continuum and line luminosities per unit area of
cloud at the inner radius when the luminosity case was used.
This was to avoid exponential limits on floating point numbers.
Today the code will report luminosities when appropriate.
This command will use the old style for backwards compatibility.
If this is used the \cdCommand{save continuum} output will be
$\nu L_{\nu} / 4 \pi r_{inner}^2$ and have units $\ergpscmps$.

\item[set save prefix ``test'']  This adds a prefix
to all filenames and so
provides an easy way to change all file names at one time.
\label{sec:CommandSetSavePrefix}
The prefix is any string that appears between the pair of
double quotes on the command line. This command should precede
all save commands.

\end{description}

\subsection{Set species gbar 0.01}
This sets a default collision strength for database transitions that have 
no collision or radiative data.
You can turn this feature off by setting the value to zero.

\subsection{Set species continuum}
\label{sec:SetSpeciesContinuum}

\par
This changes the lower and upper bounds, and the resolution,
of the species pseudo-continuum reported with the
\cdCommand{save species continuum} command,
Section \ref{sec:SaveSpeciesContinuum}.
The species must be specified in double quotations, followed
by three numbers: the short and long wavelength limits
(both in \AA ) and the number of logarithmic bins to
divide this into.
All three numbers must be specified.
The default are lower and upper wavelength limits of
\speciesConWlLo\ and \speciesConWlHi,
broken into \speciesConNbins\ bins.
For example,
%
\begin{verbatim}
set species "Fe+" continuum 3000 5000 200
\end{verbatim}
%
would accumulate the Fe~{\sc ii} emission from 3000 \AA{}
to 5000 \AA{} in 200 logarithmic bins.


\subsection{Set temperature [floor, convergence]}
\label{sec:SetTemperatureOptions}

These commands change some details of the thermal solution.
\begin{description}
\item[set temperature floor]  This sets a lowest kinetic temperature
to allow in a calculation.
When the temperature falls below this floor the code
goes over to a constant temperature solution at the floor temperature.
This provides a way to mimic having a minimum temperature that
is set by some external and unspecified source of heat.
The temperature will be interpreted as a log if it is less than
or equal to 10 or the keyword \cdCommand{log} is present, and as
linear if the keyword \cdCommand{linear} is present.

\item[set temperature convergence]  The gas kinetic temperature
is set by the balance
between the heating and cooling.
This command sets the relative error
allowed in the heating-cooling match.
The number is the largest fractional
error allowed.
It is interpreted as the error itself if it is positive
and the log of the error if it is less than or equal to zero.
The default is 0.005.
This will be the error allowed in each zone.
The total error
or energy conservation mismatch integrated over a cloud will be much smaller,
usually of order ten times smaller than the tolerance specified.
\end{description}

\subsection{Set test}

This sets the logical variable \cdVariable{lgTestCodeEnabled} to true.
It provides
the facility to conditionally run test code somewhere
in the main body of \Cloudy.

\subsection{Set trimming -9  [upper lower, off] [old, new]}

The code saves execution time by not computing the population of stages
of ionization with trivial abundances.
The thresholds for excluding an
ionization stage are chosen with photoionization equilibrium in mind.  These
may not be appropriate for some other conditions or if your definition of
trivial is different from mine.

This command changes the limit for the smallest fractional abundance,
$n(\mathrm{A}^{+x})/n(\mathrm{A})$, to consider.
The smallest relative abundance to be considered
for a stage of ionization higher than the ionization peak is changed with
the \cdCommand{upper} keyword.
The command \cdCommand{set trimming upper off} will turn off trimming
of the highest stage of ionization.
The smallest relative abundance of
ions below the peak is changed with the \cdCommand{lower} keyword.
The default relative
ionization fractions are $10^{-6}$ and $10^{-10}$ respectively.
If no keyword appears
then both are changed to the number entered.
The argument is the log of
the lowest or highest ionization fraction to consider.

Generally, line excitation energies for atoms with
stages of ionization higher than the peak ionization will
strongly exceed the ambient temperature
so these ions will have little influence on the calculated
temperature or spectrum.
This is not true for lines formed by ions below the peak
distribution.
The keyword \cdCommand{off} sets the smallest abundance to just above
the machine's floating precision limit.

The keyword \cdCommand{old} resets the algorithm used for trimming to
the legacy approach; the keyword \cdCommand{new} resets this to the
revised approach (the default), which includes some hysteresis in the
trimming controls to prevent zone-to-zone oscillations in the number
of active ionization states.

\subsection{Set tsqden 8}

The code performs an analysis of the predicted emission-line spectrum
at the end of the calculation.
This analysis will find the structural $t^2$
as well as a $t^2$ deduced from the [\oiii] and \hi\ spectrum
(\citealp{KingdonFerland1995} discuss this at length).
Such an analysis only makes sense for
densities below the critical density of the [\oiii] atom,
which is $\sim 10^5$~cm$^{-3}$.
The results of this analysis will not be printed if the density is
higher than the value of \cdVariable{tsqden}, currently $10^7 \mathrm{cm}^{-3}$.
The number on the
line is the log of the highest hydrogen density for which this analysis
will be performed.

\subsection{set update couplings every ion}

This updates the couplings between the ionization networks fully after
every ion balance solution, rather than only when all ions and the
chemical network have been calculated.  This may be necessary to allow
the solution to converge in cases where the coupling sources dominate
collisional and radiative processes, but will increase run times in
general.


\subsection{Set UTA}
\label{sec:SetUTA}

\par
This determines the UTA line data to be used.
Data for several species are available.
Most of them come from N.~Badnell's work
\citep{BadnellSeaton2003,Badnell+2005},
and they consist of data for all elements
up to Zn along all iso-sequences from
Li-like to Mg-like.
In the case of iron, these data cover the ionization stages
Fe$^{+14}$ through Fe$^{+23}$.

\par
There are two additional sets of UTA data
for lower stages of ionization of iron.
The data of Kisielius \citep{Kisielius2003,Ferland+2013}
concern Fe$^{+13}$ though Fe$^{+15}$, and they are the data of
preference, as described in \citet{CloudyReview13}, because
of their higher accuracy and greater number of emission lines.
The \citet{Gu2006} data concern Fe$^0$ through Fe$^{+15}$.
By default, the Kisielius data are used instead of the Badnell
and Gu data for the relevant ionization stages of iron.

\par
The following commands may be used to adjust the UTA data employed
in a simulation:
%
\begin{description}
\item[set UTA off] {\bf --} disable {\bf all} UTA lines.
	This is provided as a debugging tool.
	The simulation is considered unphysical when this
	command is given.

\item[set UTA Kisielius off] {\bf --} disable the \citet{Kisielius2003} data
	for Fe$^{+13}$ though Fe$^{+15}$.  They are enabled by default.
	When disabled, the Badnell data for Fe$^{+14}$ and Fe$^{+15}$,
	and the Gu data for Fe$^{+13}$ are used instead.
\end{description}

\par
To check which data set is used for each ionic species,
see \cdCommand{print UTA references}, Section~\ref{sec:PrintUTAReferences}.


\subsection{Set WeakHeatCool 0.02}

This command sets the threshold for the weakest cooling or heating source
included in output from the \cdCommand{save heating}
or \cdCommand{save cooling} commands.
The default is 0.05.
The number entered is interpreted as a
log if it is negative.

\section{Table lines ``name.dat''}

When the code is used as a subprogram for other, larger, codes,
it is possible to read in a series of lines whose intensity will
later be extracted
by first calling \cdRoutine{cdGetLineList},
then by calling \cdRoutine{cdLine}.
The procedure is
described in Part 2 of this document.
A set of lines that occur within
any of several data files is then read.
This method is used in most of
the large grid programs I use.

This command provides a quick way to confirm that the list of lines in
the line data files have valid names.
The command takes the name of a file
as its argument and checks that all lines can be found.
This command can also be used to check the standard ``\cdFilename{LineList*.dat}''
files that are part of the distribution.
This is actually a form of the \cdCommand{monitor} command since
it will announce a botched monitor if it cannot find any of the lines.
Note that all 5 characters of the keyword \cdCommand{lines} need
to be entered to avoid confusion with the keyword \cdCommand{linear}.
See the discussion
of \cdRoutine{cdGetLineList} in Part 2 for further information.

\section{Test}

This command runs a simple model to do a ``smoke test'', 
a quick sanity check on the code.
The single command produces a series of commands to run a two-zone
constant-temperature model.

\subsection{Test}
The \cdCommand{table agn} command is used with the density
and temperature constant at $10^4\ \K$.
The ionization parameter is $U = 10^{-2}$ and the
calculation stops after two zones and one iteration.
Many \cdCommand{monitor} commands
are used to verify predictions.
The commands that set up the model will
be printed if the keyword \cdCommand{print} occurs on the line.

\subsection{Test large}

When the \cdCommand{large} keyword appears the model hydrogen atom
will be set to a large number of levels.

\subsection{Test H2}

This tests the large \htwo\ molecule.

\subsection{Test molecular}

This does a simulation that is cold and nearly fully molecular.
The
can be combined with the \cdCommand{H2} keyword to test the
\htwo\ molecule in the fully molecular limit.

\section{Performance speedups}

It is possible to speed up the calculation at the loss of some physical
fidelity or stability.
This section outlines the commands that make this
possible.
Each command is described in greater detail in sections earlier
in this document.

Many of these options are included in the
\cdFilename{fast.ini} initialization file.
Using this \cdFilename{ini} file will decrease the execution time
by roughly a factor of two.
This is meant as a way to roughly explore vast ranges of parameters
or as a debugging aid.

\textbf{N.B.!!  The fidelity of the simulation is compromised!}
These turn off significant physical processes.

\subsection{Turn off minor elements}

The ionization distribution and the effects of an element on the gas emission,
cooling, and opacity are all determined self-consistently.
This includes
the first \LIMELM\ elements and many of these have trivial abundances.
The
calculation can be sped up with little loss of physical fidelity elements with
very small abundances are turned off.
This is done with the \cdCommand{element off} command.

The \cdCommand{element limit off} command described on page 
\pageref{sec:ElementLimitOffCommand}
will turn off all elements which have an
abundance smaller than a specified limit.
The following will turn off all elements with abundances less than $10^{-7}$ of hydrogen.
\begin{verbatim}
element off limit -7
\end{verbatim}

Specific elements can be turned off by giving that element on the  \cdCommand{element off} command.
The ini files \cdFilename{c84.ini},
\cdFilename{fast.ini}, and \cdFilename{ism.ini},
among others,
include a set of \cdCommand{element off} commands to do this.

Turning off an element will affect results if the opacity,
heating, or cooling
due to the element is significant.
Minor elements can have major effects
in very dense, optically thick, gas, where all coolants are thermalized and so are close to the
black body limit.

\subsection{Degrade the grain physics}

The treatment of grain physics is today's state of the art
(\citealp{VanHoof2001, VanHoof2004}) and can be CPU-intensive.
Grains have several major effects.
Elements locked up in solids are missing from the gas phase and so their
cooling is also missing \citep{KingdonFerlandFeibelman1995}.
Grains provide the dominant opacity across much
of the optical, near IR, and UV spectrum.
This absorbs the incident
continuum, affects the radiative acceleration of the gas, and the transfer
of optically thick lines.
Radiative and collisional interactions between
grains and the gas affect the gas temperature.
In PDR and molecular regions,
grains have an important effect on the charge balance of the gas.
This
affects both the ionization and chemistry (\citealp{Abel2005}).
Grain surface
reactions and molecular freeze-out change the chemistry and charge balance
of the gas.

The first effect can be included at no expense by setting the gas-phase
abundance of an element to a depleted value.
Gas-phase abundances can be
decreased by typical ISM depletion factors using the
\cdCommand{metals deplete} command.
This does not turn on grains and applying a depletion
factor has no impact on the execution time.

\cdTerm{Turn off quantum heating}.  Quantum heating
is the process where a grain's
temperature spikes after absorbing a single photon.
This temperature spiking
has an effect on the Wien tail of the grain's emitted spectral energy
distribution but no other effect.
Quantum heating can be disabled with
the \cdCommand{no qheat} option on the \cdCommand{grain} command if the thermal continuum is not
something you need to predict.
Quantum heating only affects the emitted
spectrum so turning if off will not affect results if the emitted IR
continuum does not affect the gas.
The IR continuum will affect the gas
if it is absorbed by the gas or dust, which can occur with some line, or
if the source is optically thick in the near IR continuum.

\cdTerm{Use single rather than distributed grains}.
If the \cdCommand{single} keyword appears
on the \cdCommand{grain} command then a single mean grain size
will take be used instead
of the size-resolved grain distribution.
This affects the entire simulation,
including the gas temperature and emission line spectrum,
because the grain
- gas interactions depend on the grain size.

\subsection{Turn off the chemistry}

The chemical network requires a fair amount of time due to the size of
the matrix that must be solved.
The chemical state is determined even in
ionized regions to ensure continuity in the transition
between the \hplus\ - \hO\ - \htwo\ regions.
The chemistry can be ignored if the simulation only includes ionized regions.
This is done with the \cdCommand{no molecules} command.

\subsection{Turn off level 2 lines}

This is done with the \cdCommand{no level2 lines} command.
The level 2 lines are a large set of transitions that came
from the Opacity Project.
They are important in dense or very high
ionization gas,
and for pumping excited levels (such as \ci*\ or \cii*)
of many heavy elements.

\subsection{Degrade the continuum resolution }

The resolution of the coarse continuum mesh affects the execution time
since photoionization rates and opacities involve repeated sums over it.
The resolution is changed with the \cdCommand{set continuum resolution} command.

If you change the continuum resolution then the standard stellar SEDs and
grains that are part of the code distribution will need to be recompiled.
This is done with the \cdCommand{compile grains} 
and  \cdCommand{compile stars} commands.

\subsection{Make the iso-sequence model atoms as small as possible}

The code treats several iso-sequences with a unified model atom.
The
execution time increases when the number of levels within the model atoms
is increased.
Less time will be used if the smallest possible number of
levels is used.
The number of levels is adjusted with the \cdCommand{levels} option
on the \cdCommand{database} command.
The
\cdCommand{database levels} commands for the
H-like and He-like atoms accepts the keywords \cdCommand{very small}
to make the atom
as compact as possible.
The line spectrum predicted by these atoms is likely
to be highly inaccurate,
and Case~B intensities are far better when a compact model is used.

