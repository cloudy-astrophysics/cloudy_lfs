\chapter{INTRODUCTION}
% !TEX root = hazy1.tex

\section{Overview}

\noindent \Cloudy\ is a microphysics code.
It is designed to simulate physical conditions within clouds
ranging from the intergalactic
medium to the high-density LTE and STE limits.
The range of temperatures extends from the CMB temperature
up to \TEMPLIMITHIGH\ and the physical state
ranges between fully molecular to bare nuclei.
It predicts the thermal, ionization, and chemical structure
of a cloud which lies with these limits, and
predicts its observed spectrum.

This document describes the input, output,
and assumptions for the code.
It fully defines the commands used to drive the
program.
Part 2 gives details about the program
structure, its output, the computational environment, and how to call \Cloudy\
as a sub-program of other programs.
The methods, approximations, and assumptions used by \Cloudy\ are
outlined in Part 3 of this document, although this part, like \Cloudy\ itself,
is still under construction.
The \emph{Quick Start Guide} summarizes these
three volumes and gives a general overview of the simulations.

Many environments are encountered in which dilute gas is heated and
ionized by the radiation field striking the gas.  Under these
circumstances it is possible to predict the physical conditions (that
is, the distribution of ionization, density, and temperature) across
the cloud, and its resulting spectrum, in a unique and self-consistent
manner.  This is done by simultaneously solving the equations of
statistical and thermal equilibrium, equations that balance
ionization-neutralization processes, and heating-cooling processes,
respectively.  The graduate text \citet[hereafter
AGN3]{Osterbrock2006} describe the physics governing such environments
and \citet{Ferland2003} provides additional details on current
research topics.

The instructions for downloading and
setting up the code are on the web site
\href{http://www.nublado.org}{www.nublado.org}.
We maintain a discussion board at
\href{http://tech.groups.yahoo.com/group/cloudy_simulations}
{tech.groups.yahoo.com/group/cloudy\_simulations}
where questions can be posted.

\section{What must be specified}

\noindent
One powerful asset of photoionization analysis is the large number of
observables resulting from only a few input parameters.
Intensities of a very large number of spectral lines\footnote{
There is no limit to the number of spectral lines that the code can
compute because there is no limit to the number of levels of atoms of the
H-like (\hO, \heplus, etc) and He-like (\heO, Li$^+$, etc)
isoelectronic sequences.
Processor speed and memory limit the simulation to a few million lines in
most applications with today's computers, however.}
are predicted by \Cloudy.  These result
from the specification of only a) the shape and intensity of the external
radiation field striking a cloud,
b) the chemical composition and grain content of the gas, and c) the geometry
of the gas, including its radial extent and the dependence of density on
radius.  The following subsections describe the general philosophy of the
specification of each.

\subsection{Incident radiation field}

\noindent Both the shape and intensity of the external radiation field striking the cloud
must be specified.
Often this is the region's only source of heat and
ionization.
Different commands are usually used to specify the external radiation field
shape and intensity, although a few commands specify both.

\begin{description}

\item[Shape of the external radiation field]
The shape of the spectral energy distribution (SED)
should be fully specified between a frequency
of \emmmhz\ $(\lambda \approx$ \emmcm\ -- this is the lowest
frequency observable with LOFAR) and an energy
of \egamry\ $(h\nu =$ \egamrymev)
if possible.\footnote{
In much of the following discussion photon energies will
be given in Rydbergs.  The ionization potential of hydrogen is nearly 1
Rydberg.  See the discussion in Part 2 of this document for an exact
definition and how to convert the Rydberg to other units.}
A physically-motivated radiation field spanning the full energy range
should be specified if possible.
This can be specified as a fundamental
form (such as blackbody emission, optically thin bremsstrahlung emission,
or a power law with optional exponential cutoff), interpolated from tables
of points, or as the radiation field transmitted through a cloud and
predicted by previous calculations.
Additionally, a set of built-in radiation fields
(for instance, emission from many model atmospheres, the observed
Crab Nebula spectral energy distribution (SED), or several typical AGN SEDs) can be specified as look-up tables.

\item[Intensity or luminosity of the external radiation field]
The intensity of the external radiation field striking the
cloud must be specified.
This can be given either as the flux striking a unit surface
area of cloud or as luminosity  radiated by the central object into $4\pi$~sr.
The code must be able to derive the flux of photons (cm$^{-2}$ s$^{-1}$) striking
the illuminated face of the cloud.
There are many commands that set these.
Two cases, the intensity and luminosity cases, can be used to
set the strength of the radiation field.
These are described in the next Chapter.

\item[Combining several radiation fields]
Up to 100 different radiation fields can be included as part of
the total field striking the cloud.
There must be exactly the same number of shape and luminosity
specifications.
The code will stop if there are not.

\end{description}

\subsection{Chemical Composition}

\noindent The program considers the lightest \LIMELM\ elements in detail.
Grains are
not part of the default composition but can be included.
All stages of
ionization are treated, and all published charge exchange, radiative
recombination, and dielectronic recombination processes are included as
recombination mechanisms.
Photoionization from valence and inner shells
and many excited states, as well as collisional ionization by
both thermal and supra-thermal electrons, and charge transfer, are included as ionization mechanisms.
The chemistry includes many molecules and will go to the fully
molecular limit (H in \htwo\ and C or O in CO).
Although the default composition
is solar several other standard mixtures can easily be specified and an
arbitrary composition can be entered.

\subsection{Geometry}

\noindent The geometry is always 1D spherical but can be made effectively plane
parallel by making the inner radius much larger than the thickness of the
cloud.  The default is for the gas to have constant density and to fully
fill its volume
Other pressure laws and strongly clumped clouds can be computed as well.

\Cloudy\ normally assumes an \cdTerm{open geometry},
or one in which the gas has
a very small \cdTerm{covering factor}
(these terms are defined in the following chapter
and in section 5.9 of AGN3).
This can be changed with
the \cdCommand{sphere} (see section \ref{sec:CommandSphere}) command which sets the covering factor
to a large enough value for light escaping
the cloud in the direction towards the central object to always interact
gas on the other side.
This is a \cdTerm{closed geometry}.
Line photons which
cross the central hole interact with line-absorbing gas on the far side
if \cdCommand{sphere static} is set but do not interact (because of a Doppler shift
due to expansion) if \cdCommand{sphere expanding} is set.
This case is the default when \cdCommand{sphere} is specified.

\subsection{Velocity Structure}

\noindent \Cloudy\ normally assumes that only thermal motions broaden spectral lines and that there is no internal velocity structure.
These assumptions can be changed.
A component of microturbulence can be added with the
\cdCommand{turbulence} command.
A flow can be computed with the \cdCommand{wind} command.

\section{The input deck}

\noindent \Cloudy\ is driven by a set of command lines.
These normally reside in a small input file which the code reads
when it starts.
The commands are four-letter keywords (either
upper or lower case) followed by free-format numbers that
may be mixed with letters.  The input is dependent on case for items
within quotation marks \verb|"|, such as file names (although the
case of filenames may then be ignored on some operating
systems).  Where the names of 
molecular species are specified, these will often need to be in quotation
marks, so that, for example, carbon monoxide (CO) and cobalt (Co) can be
distinguished.

When \Cloudy\ is executed as a stand-alone program standard input
(\cdFilename{stdin}) is read for input and standard output
(\cdFilename{stdout}) is used for output.  It may also be run using
commands stored in an text input file.  As an example, create a small
file (say, called \cdFilename{simple.in}) containing the following
lines:

\begin{verbatim}
title example input
hden 5 # log of hydrogen density, cm^-3
blackbody 5e4 K # spectral shape is a 50000K blackbody
# intensity of blackbody is set with
# ionization parameter so starting radius is not needed
ionization parameter -2
\end{verbatim}
Suppose that the code has been compiled to create the executable \cdFilename{cloudy.exe}.
Then, the model described by the parameters in this input file could be
computed by typing the following at the command prompt:
\begin{verbatim}
cloudy.exe -r simple
\end{verbatim}
Note that on many Linux systems the command will have to be written as
\begin{verbatim}
./cloudy.exe -r simple
\end{verbatim}
for system security.  
The file \cdFilename{simple.in} will be read for input and the output will be in the file
\cdFilename{simple.out}.  

The \href{http://www.nublado.org}{www.nublado.org} web site has many more details
about running this version of \Cloudy\ in the page
\href{http://trac.nublado.org/wiki/RunCode}{RunCode}.

It is also possible for a larger program to drive \Cloudy\ directly by
treating it as a subroutine. 
See \href{http://trac.nublado.org/wiki/RunCode}{RunCode} for more details.

\section{What is computed and reported}

The code creates an output file as the model is computed.
The program begins by echoing the input commands (after stripping hidden
comments, see Sect.~\ref{sec:CommentsInInput} for details).  Comments are
always ignored by the parser.  The input stream ends with either a blank line or
the end-of-file.  Some properties of the incident radiation field, such
as luminosity and number of photons in certain frequency ranges, are then
printed.

\Cloudy\ works by dividing a cloud into a set of thin concentric shells
referred to as zones. The zones are chosen to have thicknesses that are
small enough for the physical conditions across them to be nearly constant.
Adaptive logic continuously adjusts the physical thicknesses of these shells
to ensure this.  Typically $\sim$100 to 200 zones are computed in an optically
thick model of an H~II region.  The physical conditions in the first and
last zones are always printed and intermediate zones may be printed if needed
(this is governed by the \cdCommand{print every} command).
The output for each zone begins with a line giving the zone number, its
electron temperature, the distance from the center of the spherical nebula
to the center of the zone, and some other properties of the solution.  The
next line gives the relative contributions of various emission lines to
the radiation pressure if this amounts to more than 5\% of the gas pressure.
The remaining lines of output give the relative populations of
ionization stages of the other elements.
Many details about the conditions within the zone are
intermixed with these relative populations.

After the zone calculations are complete and the model is finished the
code will explain why the calculation stopped.  It is very important to
confirm that the calculation stopped for the intended reason.  Some warnings,
cautions, or notes about the calculation may also be printed.  The code
is designed to be autonomous and self aware.  This self checking will ensure
that its range of validity is not exceeded. It will complain if it goes
outside its range of validity, if it feels that some parameter has been
incorrectly set, or if something surprising has happened during the
calculation.  This is an essential core feature of the code since it is
now often used to generate grids of thousands of models, making it impossible
to validate individual models by hand.

The final print out begins with a recapitulation of the entered commands
followed by the predicted emission-line spectrum.  The first two columns
of the emission-line spectrum give an identifying label and wavelength for
each spectral line.  The third column is the log of the luminosity [erg
s$^{-1}$] or intensity [erg cm$^{-2}$ s$^{-1}$] of the emission line, and the last column
gives its intensity relative to the reference line.
The reference line is H$\beta$ by default and other emission lines
can be chosen with the \cdCommand{normalize} command.
The third column will be either the luminosity or intensity.
The luminosity (energy radiated by a shell of
gas covering $\Omega$ sr of the central object) is predicted if the incident radiation field is
specified as a luminosity.  The line intensity $4\pi J$ (the energy emitted per
square centimeter of the gas slab) is predicted if the incident radiation field is specified
as an intensity.  If the geometry is spherical but the incident radiation field is specified
as an intensity then the line intensities will be expressed per unit area
at the inner radius.  Only the strongest emission lines are printed; the
relative intensity of the weakest line to print is adjusted with the
\cdCommand{print faint} command.

The last pages of the output gives some averages of the ionization
fractions over the slab, the optical depths in various lines and continua,
and other properties of the nebula.

There is a vast amount of information that is computed but not given
in the standard output file.  Special output files are created with the
\cdCommand{save} commands described below.
