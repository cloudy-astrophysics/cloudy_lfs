\chapter{LIMITS, ASSUMPTIONS, AND RELIABILITY}
% !TEX root = hazy2.tex

\section{Overview}

This section outlines some of the assumptions and limits that define
the range of validity of \Cloudy.
The code is designed to be autonomous
and self-aware, and to check that these limits are not exceeded during a
calculation.
This self-checking is a central feature of the code since
it is designed to be used to compute large grids with thousands of models,
where the examination of individual results would not be possible.
\Cloudy\ should print a warning after the last zone results
if any aspects of the calculation are in doubt.

\section{Time steady}

\Cloudy\ does not assume that the gas is in equilibrium.
In most cases
it does, however, assume that atomic processes occur on timescales that
are much faster than other changes in the system,
so that atomic rates have
had time to become time-steady.
It is possible to follow time-dependent
conditions or an advective flow.

In practice most calculations assume that the cloud is old enough
for
atomic processes to have become time steady.
The \cdCommand{age} command (described
in Part I of this document) should be used to
specify the age of the cloud.
The code will then confirm that the time-steady
assumption is valid by
comparing the system's age with a host of rates and timescales,
and will
generate a warning if the environment is not time-steady.

Various time scales characterize the approach to equilibrium of an ionized
gas (see \citealp{Spitzer1962}, and \citealp{Ferland1979}
for a specific application).
Generally, for an ionized gas with nebular temperatures
$(T \sim 10^4 \K)$, the longest
is the \hplus\ recombination time scale,
\begin{equation}
T_{rec}  = {1 \over {\alpha _A \left( {T_e } \right)\,n_e }} = 7.6\,t_4^{0.8}
\;n_4^{ - 1} \;{\mathrm{years}} = 0.66\,t_4^{0.8} \;n_9^{ - 1} \;{\mathrm{hours}}%(1)
\end{equation}
where $t_4$ is the temperature in units of $10^4 \K$,
$n_9$ is the electron density in units of $10^9 \pcc$,
and case A recombination is assumed (AGN3).

The time scales are far more ponderous in molecular regions.
Generally
among the longer of the time scales is the time to form \hminus,
an important
pacesetter for \htwo\ formation in grain-free environments.
This time scale is roughly given~by
\begin{equation}
T_{molecule}  = {1 \over {\alpha _{rad} \left( {T_e } \right)\,n_e }} =
0.3\,t_3^{ - 0.8} \;n_9^{ - 1} \;{\mathrm{years}}% (2)
\end{equation}
where $t_3$ is the temperature in units of $10^3 \K$.

If the age of the cloud is not long enough for atomic processes
to become time steady then a time-dependent calculation should
be done.
If the gas flow is fast enough for advection to add significant
terms to the statistical equilibrium equations then a wind 
calculation should be performed.
Commands for these simulations are described in Chapter
\ref{Hazy1-sec:DynamicalTimeDependent},
\cdSectionTitle{\refname{Hazy1-sec:DynamicalTimeDependent}},
in Hazy 1.

\section{Atomic and molecular database}

This section outlines some of the atomic and molecular physics issues
that affect the reliability of numerical simulations of nebulae.
These
uncertainties were extensively discussed in the
Lexington Plasma 2000 meeting
(ASP Conf series 247,
\cdSectionTitle{Spectroscopic challenges of photoionized plasmas,}
Gary Ferland \& Daniel Savin, editors) and they underscore the importance
of atomic and molecular physics for the interpretation of astrophysical
spectroscopy.

\subsection{Collisional processes}

By its nature, the electron temperature of a photoionized gas is low
compared with the ionization temperature of the mixture of atoms and ions,
as defined by the Saha equation (if the two were comparable, the gas would
be collisionally ionized).
Because of this, the rate coefficients describing
collisional effects, such as the production of cooling emission lines, are
often dominated by the cross sections near threshold.  This is where
laboratory experiments are difficult and \emph{ab initio} quantum theory
must often be used.
As a result, the collision strengths undergo constant revision,
towards better and more reliable values.

To cite one extreme example,
the collision strength for transitions within
the $^3P$ ground term of Ne$^{+4}$
underwent three revisions between 1984 and 1991,
each by a factor of 10,
because of theoretical uncertainties in positions
of autoionizing states with unknown energies
\citep{Lennon1991}.  The
intensities of all emission lines can be affected by major changes in the
atomic data for only one line for some conditions.  This is because (in
this case) the infrared fine structure lines of Ne$^{+4}$ can be important
coolants in low-density high-ionization gasses such as planetary nebulae,
and changing their cooling rate alters the thermal structure of the entire
nebula.  Such changes often give even models of time-steady objects such
as planetary nebulae certain time-dependent characteristics.

At present, there are fairly reliable calculations of collision
strengths and transition probabilities for the majority of the strong
optical and ultraviolet lines in moderate ionization nebulae.  A
series of papers by Oliva and collaborators (see \citealp{Oliva1996}
and \citealp{VanHoof2000a})
outline observational evidence concerning accuracies in collision
strengths of moderate ionization far infrared lines.

Some strong cooling lines of high ionization species do not have accurate
collision rates.
As an example, few of the ``level2'' lines included in
the code have real collision strengths.
Various forms of the ``g-bar''
approximation are used for those lines that do not have accurate collision
rates.

\subsection{Photoionization cross sections}

The photoionization cross-section database has undergone a dramatic
improvement with the completion of the Opacity Project (\citealp{Seaton1987}) and
its fitting with analytic approximations (\citealp{VernerFerlandKorista1996}).
These
are the photoionization cross sections used by \Cloudy\ and
they should be
as accurate as 10\%.
All inner shell multi-electron processes are included
(\citealp{Kaastra1993}) using distorted wave cross sections (referenced
in Verner et al.).
This part of the data base is in fairly good shape,
although greater accuracy is always desired.

``Fano profiles'', due to autoionization resonances, appear as large
changes in photoionization cross section that occur over a narrow range
of energy, are averaged over, as described by Verner et al.
Their positions
are not accurately known, and they could make a difference
if sharp spectral
features occur in the stellar continuum at the position of a resonance.
The photoionization rate could be changed dramatically in this case.
Experimental data would be needed to upgrade the photoionization data base
to include Fano profiles.

\subsection{Recombination rate coefficients}

Recombination from closed shell species is accurately known (\citealp{VernerFerland1996}) since these are dominated by radiative recombination.
Reliable
dielectronic recombination coefficients do not now exist for most other
stages of ionization.
Currently there is no theory that can reproduce the
best experiments (\citealp{Savin1999}).
For these, \Cloudy\ uses the guestimates
described in Part I of this document.
This is clearly the greatest single
gap in the atomic data base today.
\citet{Savin2000} shows an example where
this uncertainty has a direct impact on cosmological studies.

\subsection{Charge transfer}

The rate coefficients for charge transfer are another uncertainty in
the atomic and molecular database.
This process is sometimes the dominant
neutralization mechanism for singly or doubly ionized heavy elements.
At
present many charge exchange rate coefficients are the result of
Landau-Zenner calculations using semi-empirical potential curves (\citealp{Kingdon1996}, \citealp{Kingdon1998b}).
These are thought to be no more accurate than
a factor of three.
Even the best quantal calculations are not thought to
have an accuracy much better than 50 percent.
Unpublished tests suggest
that these uncertainties affect some line intensities at the
$\sim$20\% level,
and a few by more than this.

\section{Continuous opacity}

All significant continuous opacity sources are treated for the energy
range considered by the code,
\emm\ to \egamry .
These opacity
sources include inverse bremsstrahlung, grains (when present),
\hminus\ absorption,
electron scattering, the damping wings of strong resonance lines (i.e.,
Rayleigh scattering), pair production,
photoelectric absorption by the ground
and excited states of all ions of the lightest \LIMELM\ elements,
and photoabsorption by molecules.
This treatment should be adequate as long
as the optical depths to electron scattering are not large.
\Cloudy\ is not
now designed to simulate Compton-thick regimes.
(A warning will be issued
after the last zone calculation if the nebula is very optically thick to
electron scattering.)

\section{Temperature range}

\Cloudy\ assumes that the electrons are non-relativistic, which limits
it to temperatures below roughly 10$^9$~K.
\Cloudy\ goes to the Compton
temperature of the radiation field to great accuracy in the limit of very
high levels of ionization for blackbody radiation fields with temperatures
between \TEMPLIMITLOW\ and \TEMPLIMITHIGH.
There is no formal lower temperature limit to
its validity.
Note that very cold gas is rarely in steady state, however,
because of the very slow collision rates.
Similarly, the collision
timescales in very hot gas $(T \gg 10^8 \K)$ are not rapid enough
to ensure that
the electrons and ions have the same temperature,
or that heating - cooling
balance has become time steady (\citealp{Johnson2007}).
Electron-ion
decoupling is not now included.

The range of validity of the code is approximately from 10~K to 10$^9$~K.
Temperatures outside this range can still be treated,
although with greater
uncertainty.
The code will not permit temperatures below \TEMPLIMITLOW\ or above
\TEMPLIMITHIGH.

\section{Density range}

There is no formal lower limit to the density that \Cloudy\ can treat.
The set of heavy element fine structure lines, which dominate cooling at
low densities, is complete for astrophysically abundant elements, and fine
structure line optical depths, continuum pumping, and maser effects are
fully treated.

There is no formal high-density limit, although the simulation is less
complete at high $(n \cong 10^{10} \pcc$ densities.
Line radiative transfer processes are fully treated with 
either escape or Sobolev approximations 
(see \citealp{Kalkofen1987} and  \citealp{Avrett1988})
and collisional-radiative ionization
processes for excited levels of the heavy elements are treated.
All species of H-like
and He-like isoelectronic sequences are treated as many-level atoms,
including all of the physical processes that allow the approach
to LTE (see, for example, \citealp{Mihalas1978}).
Tests with a hydrogen density of $10^{19} \pcc$
show that the hydrogen and helium atoms and the hydrogen molecules go to
LTE at high densities.
They go to strict thermodynamic equilibrium when
exposed to a true black body.
The treatment of Stark broadening for hydrogen
lines follows \citet{Puetter1981},
so line radiative transfer is treated correctly
(in the context of the escape probability formalism) for densities above
$\sim10^{10} \pcc$.

The treatment of the other 28 isoelectronic sequences is presently not
as complete as the H and He-like sequences.
Three-body recombination is
included as a general recombination process, so the treatment of these
elements is approximately correct at high densities.

\Cloudy\ has been tested at densities of $10^{-8} \pcc$
and $10^{19} \pcc$ on 32-bit machines.
The numerical (not physical) limit to the density will actually
be set by the limits to the range of the floating point numbers allowed
by the machine in use.
The physics incorporated in the code imposes no
lower limit to the density.
The physical high-density limit is roughly
$10^{15} \pcc$, and is set by the approximate treatment
of three-body recombination - collisional ionization for the
heavy elements.

\section{Radiative transfer}

Several tests presented in the test suite show that the continuum transfer
methods are in excellent agreement with known exact results.

Line intensities are predicted with stellar atmosphere conditions in
mind.
Radiative transfer effects, including continuum pumping and possible
maser emission, are treated.
Nebular approximations, such as the
approximation that all atoms are in the ground state, are not made.
Collisional effects, including excitation and de-excitation, continuum
fluorescence, recombination, etc, are all included as general line
excitation mechanisms.
The treatment of level populations is designed to go to LTE
in the high particle or photon density cases.

The transfer of subordinate lines, those where both upper and lower
level of the transition occur in excited state, of the H-like and
He-like iso sequences at high ($n \gg 10^{10} \pcc$) densities
is a concern.
Resonance lines should be fine.

\section{Reliability}

Several issues affect the general question of the reliability of the
code.
The first is the effects of the bugs that surely must exist in a
code the size of \Cloudy.
I have seldom found bugs in sections of the code
older than roughly two to three years.
Younger sections of the code
sometimes contain bugs that only manifest themselves in exceptional
situations.
The issue of reliability in the face of complexity will
increasingly be the single major problem limiting the development of
large-scale numerical simulations (\citealp{Ferland2001b}).
New methods of writing
code will have to be developed if we are to take full advantage of the power
of future machines.
Machines are getting faster more quickly than people
are getting smarter.

The second issue is the validity of the numerical methods used to simulate
conditions in the plasma.
Fundamental uncertainties arise for cases where
the density is high $(n\gg 10^{10} \pcc$).
The line radiative transfer techniques
used by \Cloudy\ are approximate (see the discussion by \citealp{Avrett1988}).
Unfortunately, no definitive calculation now exists for the complete
non-LTE equilibrium and line emission for an intermediate density
$\sim10^{13} \pcc$ cloud.
For less extreme conditions $(n < 10^{10} \pcc)$ nebular
approximations are valid.

Test cases that are designed to exercise the code in well-posed limits
and for certain standard nebulae are in the test suite.
The code is well
behaved and agrees with predictions of similar codes in these limits.
The
discussion presented in \citet{Ferland1995} and \citet{Pequignot2001}
suggests that 10\% accuracy can be reached for the intensities of the
stronger lines.

The uncertainties can probably best be judged by looking at both the
dispersions among the various photoionization calculations presented in
\citet{Ferland1995} and \citet{Pequignot2001}, and by the changes
in the test suite.
Much of the dispersion is due to improvements in the
atomic database.

There can be little better way to close a discussion of reliability
than to quote the warning included in the description by
\cite[see page xiii]{Kurucz1970} of ATLAS5, a code more than an order of
magnitude smaller than \Cloudy:

\begin{quote}
\cdTerm{WARNING}

\noindent ``There is no way to guarantee that ATLAS5 does not contain errors.
In fact, it is almost certain that it does, since the code is so long.\dots
We also point out that the computation of a model atmosphere
should be considered a physical experiment.  The program may not be able
to calculate a model for conditions that do not occur in real stars or for
conditions that violate the initial assumptions on which the program is
based.''
\end{quote}

\section{The future}

The eventual goal is for \Cloudy\ to give reliable results
for all extremes
of conditions between and including the intergalactic medium and stellar
atmospheres.
I estimate that the code is now well over halfway complete.

Current work centers on taking advantage of parallel machines. Once
finished, we will be able to use this additional computational power to
improve the treatment of hydrodynamics, line radiative transfer, and
higher-order dimensionality.
